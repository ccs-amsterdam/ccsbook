\section{Calculating values}
\label{sec:calculate}

Very often, we need to calculate values for new columns or change the content of existing columns.
For example, we might wish to calculate the difference between two columns,
or we may need to clean a column by correcting clerical errors or converting between data types.

In these steps, the general pattern is that a column is assigned a new value based on
a calculation that generally involves other columns.
In both R and Python, there are two general ways to accomplish this.
First, you can simply assign to an existing or new column,
using the column selection notation discussed in \refsec{datatypes}:
\verb+df["column"] = ...+ in Python, or \verb+df$column = ...+ in R.

Both Python and R also offer a function that allows multiple columns to be changed,
returning a new copy of the data frame rather than changing the original data frame.
In R, this is done using the \pkg{tidyverse} function \fn{mutate}, which is the recommended way to compute values.
The Python equivalent, \pkg{pandas} function \fn{assign}, is used more rarely as it does not offer many advantages over direct assignment.

In either case, you can use arithmetic: e.g. \verb|rep - dem| for computes the difference between these columns.
This works directly in R \fn{mutate},
but in Python or in R direct assignment you also need to specify the name of the dataframe.
In Python, this would be \verb+d.rep - d.dem+, while in R this is \verb+d$rep - d$dem+. 

In many cases, however, you want to use various functions to perform tasks like cleaning and data conversion
(see \refsec{function} for more explanation of built-in and custom functions).
For example, to convert a column to numeric you would use the base R function \fn{as.numeric} in R or the \pkg{pandas} function \fn{to\_numeric} in Python.
Both functions take a column as argument and convert it to a numeric column.

Almost all R functions work on whole columns like that.
In Python, however, many functions work on individual values rather than columns.
To apply a function on each element of a column, you can use \verb+df.col.apply(my_function)+
(where \texttt{df} and \texttt{col} are the names of your dataframe and column). 
There are more built-in functions that you can call on a column, for example \verb+df.col.fillna+ replaces
missing values, and \verb+df.col.str.replace+ does a find and replace.
Note that these functions (like most functions in \pkg{pandas}) work on whole columns,
so there is no need to \verb+apply+ them. 

%TODO should we give or point to some overview of possible functions?

\pyrex[output=r,format=table,caption=Mutate]{ch_data_wrangling/mutate}

To illustrate these possibilities, \refex{mutate} has code for cleaning a version of the Guns Polls
in which we intentionally introduced two problems: we added some typos to the \emph{rep} column
and introduced a missing value in the \emph{Support} column.
To clean this, we perform three steps: First, we remove all non-numeric characters using a regular expression
(see \refsec{regex} for more information on text handling and regular expressions).
Next, we need to explicitly convert the resulting column into a numeric column so we can later use it in calculations.
Finally, we replace the missing value by the column mean
(of course, it is doubtful that that is the best strategy for imputing missing values here,
we do it mainly to show how one can deal with missing values technically. You will find some more discussion about missing values in \refsec{simpleeda}).

The cleaning process is actually performed twice: lines 5-10 use direct assignment,
while lines 12-18 use the \fn{mutate}/\fn{assign} function.
Finally, lines 20-26 show how you can define and apply a custom function to combine the first two cleaning steps.
This can be quite useful if you use the same cleaning steps in multiple places,
since it reduces the repetition of code and hence the possibility of introducing bugs or inconsistencies. 

Note that all these versions work fine and produces the same result.
In the end, it is up to the researcher to determine which feels most natural in the circumstances.
As noted above, in R we would generally prefer \fn{mutate} over direct assignment,
mostly because it fits nicely into the \pkg{tidyverse} workflow and you do not need to repeat the data frame name.
In Python, we would generally prefer the direct assignment, unless it is desired to make a copy of the data
with the changes made, in which case \fn{assign} can be more useful. 



