\section{Grouping and aggregating}

The functions we used to change data above operated on individual rows.
Sometimes, however, we wish to compute summary statistics of groups of rows.
This essentially shifts the unit of analysis to a higher level of abstraction.
For example, we could compute per-school statistics from a data file containing information per student;
or we could compute the average number of mentions of a politician per day from a file containing information per articles.

In data analysis, this is called \emph{aggregation}.
In both Python and R, it consists of two steps:
First, you define which rows are \emph{grouped} together to form a new unit
by specifying which column identifies these groups.
In the previous examples, this would be the school name or ID or the date of each article.
It is also possible to group by multiple columns, for example to compute the average per day per news source.

The next step is to specify one or more summary (or \emph{aggregate}) functions to be computed over the desired value columns.
These functions compute a summary value, like the mean, sum, or standard deviation, over all the values belonging to each group.
In the example, to compute average test scores per school we would apply the average (or mean) function to the test score value column.
In general, you can use multiple functions (e.g.  mean and variance) and multiple columns (e.g. mean test score and mean parental income).

The resulting dataset is reduced both in rows and in columns.
Each row now represents a group of previuos cases (e.g. school or date),
and the columns are now only the grouping columns and the computed summary scores.

\refex{aggregate} shows the code in R and Python to define groups and compute summary values.
First, we group by poll \emph{question}; and for each question, we compute the average and standard deviation.
The syntax is a little different for R and Python, but the idea is the same:
first we create a new variable \verb+groups+ that stores the grouping information,
and then we create the aggregate statistics.
In this example, we do not store the result of the computation, but print it on the screen.
To store the results, simply assign it to a new object as normal.

\pyrex[caption=Aggregation,output=r,format=table]{ch_data_wrangling/aggregate}

In R, you use the \pkg{dplyr} function \fn{group\_by}  to define the groups,
and then call the function \fn{summarize~} to compute summary values by specifying
\verb+name=function(value)+.

In Python, the grouping step is quite similar.
In the summarization step, however, you specify which summaries to compute in a dictionary%
\footnote{See \refsec{datatypes} for more information on working with dictionaries}.
The keys of the dictionary lists the value columns to compute summaries of,
and the values contain the summary functions to apply, so \verb+{'value': [function]}+.

\subsection{Combining multiple operations}

In the examples above, each line of code (often called a \emph{statement}) contained a single operation, generally a function call.
The general shape of each line in R was \verb+data = function(data, arguments)+, that is, the data is provided as the first argument to the function.
In Python, you specify the object on which to call the function with a period,
i.e. \verb+object.function(arguments)+.

Although there is nothing wrong with limiting each line to a single operation, both languages allow multiple operations to be chained together.
Especially for grouping and summarizing, it can make sense to link these operations together as they can be thought of as a single `data wrangling' step.

In Python, this can be achieved by adding the second \verb+.function()+ directly to the end of the first statement.
Essentially, this calls the second function on the result of the first function: \verb+data = data.function1(arguments).function2(arguments)+.

In R, the data is of course included in the function arguments, so an alternative method is needed to chain function calls.
This is done using the \emph{pipe operator} (\verb+%>%+) from the (cutely named) \pkg{magrittr} package.
The pipe operator inserts the result of the first function as the first argument of the second function.
More technically, \verb+f1(d) %>% f2()+ is equivalent to \verb+f2(f1(d))+.
This can be used to chain multiple commands together, e.g. \verb+data = data %>% function1(arguments) %>% function2(arguments)+.

\pyrex[caption=Combining multiple functions,output=r,format=table]{ch_data_wrangling/aggregate2}{

\refex{aggregate2} shows the same operation as above, but chained into a single statement.


\subsection{Adding summary values}

Rather than reducing a data frame to contain only the group-level information,
it is sometimes desirable to add the summary values to the original data.
For example, if we add the average score per school to the student-level data,
we can then determine whether individual students outperform the school average.

Of course, the summary scores are the same for all rows in the same group:
all students in the same school have the same school average.
So, these values will be repeated for these rows, essentially
mixing individual and group level variables in the same data frame.

\pyrex[caption=Adding summary values to individual cases,output=r,format=table]{ch_data_wrangling/transform}

\refex{transform} shows how this can be achieved in Python and R,
computing the mean support per question and then calculating how each poll deviates from this mean. 

In R, the code is very similar to \refex{aggregate2} above, simply
replacing the \pkg{dplyr} function \fn{summary} by the function \fn{mutate} discussed above.
In this function you can mix summary functions and regular functions, as shown in the example:
first the mean per group is calculated, followed by the deviation of this mean.

The Python code also uses the same syntax used for computing new columns.
The first line selects the \emph{Support} column on the grouped dataset,
and then calls the \pkg{pandas} function \fn{transform} on that column to compute the mean per group,
adding it as a new column by assigning it to the column name.
The second line uses the regular assignment syntax to create the deviation based on the support and calculated mean. 
