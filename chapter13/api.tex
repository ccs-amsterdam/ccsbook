\section{Using APIs}

The data used in the examples above were gathered using the APIs of Twitter and Open Street Map.
\refchap{scraping} will go into this in more detail, but it is shown in \refex{funapi} here as well for the sake of completeness
(and in case you are eager to have fun with those APIs).

Twitter is a popular data source for scientific analysis, mostly because it is relatively open, public, and easy to collect.
In R, we use \pkg{rtweet}, which will automatically ask you to log in to Twitter in a browser so it can obtain an API key
(remember that you might have to \verb|install.packages("rtweet")| if you haven't already done so).
After this, you can use the code in the second part of the example to lookup users and download tweets.
Note that this will only work if you run RStudio locally: if you use a jupyter notebook (e.g. in colab)
it will not be able to connect to your browser and you will have to obtain an API key yourself,
see \refchap{scraping} for more details. 

For the geocoding, we used Open Street Maps since it can be used without an API key.
In R, you need to install \pkg{tmaptools} and then you can use the \fn{geocode\_OSM} function as shown in \refex{funtmap}.
If you want to combine the user data with the tweets as in the examples above,
you can use the code in the last part of the example. 
Don't worry if you don't understand all of the code in that snippet yet,
after reading \refchap{datawrangling} it should be clear. 



\begin{ccsexample}
  \codex[caption=Accessing Twitter]{chapter02/funrtweet.r}
  \codex[caption=Geocoding]{chapter02/funtmap.r}
  \codex[caption=Preparing user data]{chapter02/fungetgeo.r}
  \caption{Code for querying the twitter and OSM APIs}\label{ex:funapi}
  
\end{ccsexample}
