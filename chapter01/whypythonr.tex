

\section{Why Python and/or R?}
By far most work in the computational social sciences is done using
Python and/or R. Sure, for some specific tasks there are standalone
programs that are occasionally used; and there are some useful applications
written in other languages such as C or Java. But we believe it is
fair to say that it is very hard to delve into the computational analysis
of communication without learning at least either Python or R, and
preferrably even both of them.
There are very few tasks that you cannot do with at least one of them.

Some people have strong beliefs which language is ``better'' -- we do
not belong to them. Most techniques that are relevant to us can be
done in either language, and personal preference is a big factor. R
started out as a statistical programming environment, and that
heritage is still visible, for instance in the strong emphasis on
vectors, factors, et cetera, or the possibility to estimate complex
statistical models in just one line of code. Python started out as a
general-purpose programming language, which means that some things we
do feel a bit more `low-level' -- Python abstracts away less of the
underlying programming concepts than R does. This sometimes gives us
more flexibility -- at the cost of being more wordy.
In the last years, however, Python and R have been
growing closer to each other: With modules like \pkg{pandas} and
\pkg{statsmodels}, Python now has R-like functionality handling data
frames and estimating common statistical models on them; and with
packages such as \pkg{quanteda}, handling of text -- traditionally a
strong domain of Python -- has become more accessible in R.

This is the main reason why we decided to write this ``bi-lingual''
book. We wanted to teach techniques for the computational analysis of
communication, without enforcing a specific implementation. We hope
that the reader learns from our book, say, how to transform a text
into features and how to choose an appropriate machine learning model,
but find it of less importance in which language this happens.

Yet, sometimes, there are good reasons to choose one language above
the other. For instance, many machine learning models in the popular \pkg{caret} package in R under the
hood create a dense matrix, which severly limits the amount of
documents and features one can use; also, some complex web scraping
tasks are maybe easier to realize in Python. On the other hand, R's
data wrangling and visualization techniques in the \pkg{tidyverse}
environment are known for their user-friendliness and quality.  In the
rare cases where we believe that R or Python is clearly superior for a
given task, we indicate so; for the rest, we believe that it is up to
the reader to choose.

