
\section{How to use this book}

This book differs from more technically oriented books on the one hand
and more conceptual books on the other hand. We do cover the technical
background that is necessary to understand what is going on, but we
keep both computer science concepts and mathematical concepts to a
minimum. For instance, if we had written a more technical book about
Programming in Python, we would have introduced rather early and in
detail to concepts such as classes, inheritence, and instances of
classes. Instead, we decided to give such information only as
additional background where necessary and focus, rather pragmatically,
on the application of techniques for the computational analysis of
communication. Vice versa, if we had written a more conceptual book on
new methods in our field, we would have given more emphasis to
epistemological aspects, and had skipped the programming examples,
which are now at the core of this book.

We do not expect much prior knowledge from the readers of this
book. Sure, some affinity with computers help, but there is no strict
requirement on what you need to know. Also in terms of statistics, it
has helped if you have heard of terms such as correlation or
regression analysis, but even if your knoweldge here is rather
limited, you should be able to follow along. Also here, a bit more
previous knowledge helps, but you can also acquire it along the way.

This also means that you may be able to skip chapters. For instance,
if you already work with R and/or Python, you may not need our
detailed instructions at the beginning. Still, the book follows a
logical order in which chapters build on previous ones. For instance,
when explaining supervised machine learning on textual data, we expect
you to be familiar with previous chapters that deal with machine
learning in general, or with the handling of textual data.

This book is designed in such a way that it can be used as a text book
for introductory courses on the computational analysis of
communications. Often, such courses will be on the gradutate level,
but it is equally possible to use this book in an undergraduate
course; maybe skipping some parts that may go too deep. All code
examples are not only printed in this book, but also available
online. Students as well as social-scientists who want to brush up
their skillset should therefore also be able to use this book for
self-study, without a formal course around it. Lastly, this book can
also be a reference for readers asking themselves: ``How do I again
have to do this?''. In particular, if the main language you work in is
R, you can look up how to do similar things in Python and vice versa.

\begin{feature}\textbf{Code examples}
Regardless of the context in which you use this book, one thing is for sure:
The only way to learn computational analysis methods is by practicing and playing around.
For this reason, the code examples are probably the most important part of the book.
%For almost all examples we provide the code in R as well as Python.
%You can choose to use either of those languages,
%but you can also compare how the same thing is done in different languages,
%or use your knowledge of one language to improve your understanding of the other.
Where possible, the examples use real world data that is freely available on the Internet.
To make sure that the examples still work in five years' time,
we generally provide a copy of this data on the book website,
but we also provide a link to the original source.

One thing to note is that to avoid unnessecary repetition
the examples are sometimes designed to continue on earlier
snipets from that chapter.
So, if you seem to be missing a data set, or if some package is not imported yet,
make sure you run all the code examples from that chapter.

Note that although it is possible to copy-paste the code from the website accompaying this book\footnote{https://cssbook.net},
we would actually recommend typing the examples yourself.
That way, you are more conscious about the commands you are using and you are adding them to your `muscle memory'.

Finally, realize that the code examples in this book are just examples.
There's often more ways to do something, and our way is not necessarily the only good (or even the best) way.
So, after you get an example to work, spend some time to play around with it:
try different options, maybe try it on your own data, or try to achieve the same result in a different way.
The most important thing to remember is: you can't break anything!
So just go ahead, have fun, and if nothing works anymore you can always start over from the code example from the book. 
\end{feature}

