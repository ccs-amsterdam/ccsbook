Why write another methods textbook? Aren't there enough textbooks already? And what about all the great online resources?  We have been teaching computational analysis of communication for years for various universities and other organizations.  These courses used different formats, different scope from semester-long courses to short workshops, different techniques, and different levels -- but we never found the book that really fit our audience.  Regularly, students and colleagues ask us for book recommendations, and educators and administrators want to know which book to put on a reading list.  And regularly, our answer was one along the lines of: Well, there is this great book on [R/Python/Neural Networks/\ldots], but \ldots.

The ``but'', in almost all cases, has to do with the audience: students of the social sciences who have at least some knowledge of and are interested in empirical research and quantitative methods, but have no experience in programming. They do want (or have to) learn programming to conduct the analyses they are interested in, but are not necessarily interested in programming for its own sake.  They do not want to just push a button in some tool that limits their possibilities to what someone else has designed, but they also do not want to follow a whole Introduction to Computer Science with a comprehensive overview of programming concepts and paradigms that they might never need.

For years, we have therefore used our own materials to find a balance between teaching programming concepts where necessary but focussing on their application for answering questions that are of genuine interest to those studying various forms of communication.  This book is our attempt to bring together and systematize this approach to teaching the analysis of communication.

A second driver for writing this book was to get over the ``language war'' that is sometimes visible in the field. In our own research and teaching, we find both R and Python to be great tools, each with their strengths and weaknesses. Too often, existing teaching materials focus on the language rather than the underlying concept. We believe that a good computational methods textbook should give practical instructions on the implementation of a concept in a given language, but put the concept rather than the language at the forefront.  For that reason, we decided to use R and Python side by side, allowing students (and professors) to choose either -- and to allow interested readers to view the differences and similarities between the languages.

Writing this book has also been an exercise in planning and coordination. With two of us being located in Amsterdam and one in Salamanca, we had many video calls to divide tasks and discuss each other's drafts. One can hardly call us tech-adverse, but nothing is as productive (and nice) as sitting together in a room, as we experienced during a writing weekend on the island of Texel.  The COVID-19 pandemic, though, cancelled all plans for further in-person writing, and with suddenly many other unexpected priorities emerging, it took more time -- and many more online meetings -- for the final version of the book to see the light of the world.

This book would not have been possible without the continuous input we got over years -- from students, colleagues, and others.  They shaped our ideas on both how to analyze communication computationally, but also our ideas about how to teach this. It would also not have been possible without the patience of Nel, Rodrigo, and Sanne, when we again had to spend more hours than we thought on what at one point only became known as ``the book''.


