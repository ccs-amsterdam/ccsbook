\section{Statistical modeling and prediction}
\label{sec:prediction}
Machine learning, many people joke, is nothing else than a fancy name for statistics.
And, in fact, there is some truth to this:
If you say ``logistic regression'', this will sound familiar to both statisticians
and machine learning practicioners.
Hence, it does not make much sense to distinguish between statistics on the one hand
and machine learning on the other hand.

Still, there are some differences between traditional statistical approaches that
you may have learned about in your statistics classes and the machine learning
approach, even if they use the same mathematical tools. One may say that that the
focus is a different one, and the objective we want to achieve may differ.

Let us illustrate this with an example.
\texttt{mediause.csv}\footnote{You can download the file from \url{http://cssbook.nl/d/mediause.csv}}
contains a few columns from survey data on how many days per week respondents turn to different media
types in order to follow the news\footnote{For a detailed description of the
dataset, see \citet{Trilling2013phd}.}. It also contains their
age (in years), their gender (coded as female = 0, male = 1), and their education
(on a 5-point scale).

A straightforward question to ask is how in howfar the sociodemographic
characteristics of the respondents explain their media use.
Social scientists would typically approach this question by running a regression
analysis.
Such an analysis tell us how some independent variables $x_1, x_2 \ldots x_n$
can explain $y$.
In an ordinary least square regression (OLS), we would estimate
$y=\beta_0 + \beta_1 x_1 + \beta_2 x_2 + \ldots + \beta_n x_n$.

In a typical social-science paper, we would then interpret the coefficients
that we estimated, and say something like: When $x_1$ increases by one unit,
$y$ increases by $\beta_1$.
We sometimes call this ``the effect of $x_1$ on $y$ (even though, of course,
it depends on the study design wether the relationship can really be interpreted
as a causal effect).
Additionally, we might look at the explained variance $R^2$, to assess how good
the model fits our data.

\pyex{chapter09/snippets/ols}{OLS regression and full model output in Python}
\rex{chapter09/snippets/ols}{Running an OLS regression (here, without full output) in R}

Most traditional social-scientific analyses stop after reporting and interpreting
the coefficients of such a model.

But we can go a step further. Given that we have already estimated our
regression equation, why not use it to do some \emph{prediction}?

We have just estimated that

$\textrm{newspaperreading} = -0.0896 + 0.0676 \cdot \textrm{age} + 0.1767 \cdot \textrm{gender}$

By just filling in the values for a 20 year old man, or a 40 year old woman,
we can easily calculate the expected number of days such a person reads
the newspaper per week, \emph{even if no such person exists in the original dataset}.

We learn that

$\hat{y}_{man20} = -0.0896 + 0.0676 \cdot 20 + 1 \cdot 0.1767 = 1.4391$

$\hat{y}_{woman40} = -0.0896 + 0.0676 \cdot 40 + 0 \cdot 0.1767 = 2.6144$

This was easy to do by hand, but of course, we could do this automatically for a
large and essentially unlimited number of cases.
This could be as simple as shown in \refex{olsprediction}.

\rpyex{chapter09/snippets/olspredict}{Using the OLS model we estimated before to predict the dependent variable for new data, where the dependent variable is unknown.}

In doing so, we shift our attention from the interpretation of coefficients to
the prediction of the dependent variable for new, unknown cases. We do not
care about the coefficents per se, we just need them for our prediction.
In fact, in many machine learning models, we will have so many of them that
we do not even bother to report them.

As you see, this implies that we proceed in two steps: First, we use some data
to estimate our model. Second, we use that model to make predictions.

We used an OLS regression for our first example, because it is very straightforward
to interpret, and most of our readers will be familiar with it.
However, a model can take the form of \emph{any} function, as long as it takes
some characteristics (or ``features'') of the cases (in this case, people) as
input and return a prediction.

Using such a simple OLS regression approach for prediction, as we did in our
example, can come with a couple of problems, though.
For instance, even though we know that the output should be something between
0 and 7 (as that's the number of days in a week), but our model will happily
predict that once a man reaches the age of 105 (rare, but not impossible), he
will read a newspaper on 7.185 out of 7 days; and a one year old girl will
even have a negative amount of newspaper reading.
Also, more in general, it is quite an assumption to make that the relationship
between these variables are linear -- we will therefore discuss multiple
models that do not make such assumptions later in this chapter.
And, finally, in many use cases, we are actually not interested in getting
an accurate prediction of a continous number, but rather in predicting a
category. We may want to predict whether a tweet goes viral or not, whether
a user comment is likely to contain offensive language or not, wheter an article
is more likely to be about politics, sports, economy, or lifestyle.
In machine learning terms, these tasks are known as \emph{classification}.

In the next section, we will outline key terms and concepts in machine learning. After that, we will then discuss specific models that you can use for different use cases.
