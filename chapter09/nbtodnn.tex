\section{Classical Machine Learning: From Na\"{i}ve Bayes to Neural Networks}
\label{sec:nb2dnn}
To do supervised machine learning, we can use several models, all of
which have different advantages and disadvantages, and are more useful
for some use cases than for others.

We limit ourselves to the most common ones in this chapter. The
website of scikit-learn (\url{http://www.scikit-learn.org}) gives a
good overview of more alternatives.

\subsection{Na\"ive Bayes}
\label{subsec:Na\"ive Bayes}

The Na\"ive Bayes classifier is a very simple classifier that is often
used as a ``baseline''. Before estimating more complicated and
resource-intensive models, it is a good idea to estimate a simpler
model first, to assess how much better the other model actually
is. Sometimes, the simple model might even be just fine.

The Na\"ive Bayes classifier allows you to predict a binary outcome,
such as: ``Is this message spam or not?'', ``Is this article about
politics or not?'', ``Will this go viral or not?''.  It, in fact, also
allows you to do the same with more than one category, and both the
Python and the R implementation will happily let you train a Na\"ive
Bayes classifier on nominal data, such as whether an article is about
politcs, sports, the economy, or something different.

For the sake of simplicity, we will discuss a binary example, though.

As its name suggests, a Na\"ive Bayes classifier is based on Bayes'
theorem, and it is ``na\"ive''.  It may sound a bit weird to call a
model ``na\"ive'', but what it actually means is not so much that it
is stupid, but that it makes very far-reaching assumptions about the
data (hence, it is na\"ive). Specifically, it assumes that all
features are independent from each other.  Of course, that is hardly
ever the case -- for instance, in a survey data set, while age and
gender indeed are independent from each other (unless, for
instance, a war swept away a whole generation of males but not
females), this is not the case for education, political interest,
media use, and so on.  And in textual data, whether a word $W_1$ is used
is not independent from the use of word $W_2$ --- after all, both are not
randomly drawn from a dictionary, but depends on the topic of the text
(and other things).  Astonishingly, even though these assumptions are
regularly violated, the Na\"ive Bayes classifier works reasonably well
in practice.

The Bayes part of the Na\"ive Bayes classifier comes from the fact
that it uses Bayes' formula, $$ P(A \mid B) = \frac{P(B \mid A) \cdot P(A)}{P(B)} $$.

As a short refresher: The $P(A \mid B)$ can be read as: the
probability of A, given B. Or: the probability of A if B is the
case/present/true.  Applied to our problem, this means that we are
interested in estimating the probability of an item having a label,
given a set of features:
$$ P(label \mid features) = \frac{P(features \mid label) \cdot P(label)}{P(features)} $$.

$P(label)$ can be easily calculated: It's just the fraction of all
cases with the label we are interested in.  Because we assume that our
features are independent (remember, the ``na\"ive'' part), we can
calculate $P(features)$ and $P(features\mid label)$ by just
multiplying the probabilities of each individual feature.  Let's
assume we have three features, $x_1, x_2, x_3$.  We now simply
calculate the percentage of \emph{all} cases that contain these
features, $P(x_1), P(x_2)$ and $P(x_3)$.  Then we do the same for the
conditional probabilities and calculate the percentage of cases
\emph{with our label} that contain these features, $P(x_1\mid label),
P(x_2\mid label)$ and $P(x_3\mid label)$.

If we fill this in our formula, we get:


$$ P(label \mid features) = \frac{P(x_1 \mid label) \cdot P(x_2 \mid label)\ \cdot P(x_3 \mid label) \cdot P(label)}{P(x_1) \cdot P(x_2) \cdot P(x_3)}$$.

Remember that all we need to do to calculate this formula is: (1)
counting how many cases we have in total; (2) counting how many cases
have our label; (3) counting how many cases out of [1] have feature x;
(4) counting how many cases out of [2] have feature x.  As you can
imagine, doing this does not take much time to do, which makes the
Na\"ive Bayes classifier such a fast and efficient choice.  This may
in particular be true if you have a lot of features (i.e.,
high-dimensional data).

Counting whether a feature is present or not, of course, is only
possible for binary data. We could for example simply check whether a
given word is present in a text or not.  But what if our features are
continous data, such as the number of times the word is present?  We
could dichotomize it, but that would throw away information.  So, what
we do instead, is that we estimate $P(x_i)$ using a distribution, for
example a Gaussian, Bernoulli, or multinomial distribution. The core
idea, though, stays the same.

Our examples in \refsec{principles} illustrate how to do train a Na\"ive Bayes classifier.
We first create the labels (whether someone uses online news at all or
not), split our data into a training and a test dataset (here, we use
80\% for training and 20\% for testing) (\refex{preparedata}), then
fit (train) a classifier (\refex{nb}), before we assess how well it
predicts our training data (\refex{classificationreport}).

In \refsec{validation}, we discuss in more detail how to
evaluate different classifiers, but let's have a sneak preview at the most used 
measures of how well our classifier performs.

First, the confusion matrix from \refex{classificationreport} tells us how many non-users were indeed
classified as non users (59), and how many (wrongly) as users
(102).\footnote{These are the values from the Python example, the R
  example slightly differs, amongst other things due to different
  sampling.} That doesn't look very good; but on the other hand, 210
of the true users were correctly classified as such, and only 42 were
not.

More formally, we can express this using precision and recall. When we
are interested in finding true users, we get a precision of .67
($\frac{212}{212+106}$) and a recall of .84 ($\frac{212}{212+40}$).
However, if we want to know how good we are in identifying those who
do \emph{not} use online news, we do -- as we saw in the confusion
matrix -- considerably worse: precision and recall are .58 and .34,
respectively.



\subsection{Regression}
\label{subsec:regression}

Regression analysis does not make as strong an assumption about the
independence of features as the Na\"ive Bayes classifier does.  Sure,
we have been warned about the dangers of multicollinearity in
statistics classes, but correlation between features (for which
multicollinearity is a fancy term) affects the coefficients and their
$p$ values, but not the predictions of the model as a whole.  To put
it differently, in regression models, we do not estimate the
probability of a label given a feature, independent of all the other
features, but are able to ``control for'' their influence.  In theory,
this should make our models better, and also in practice, this
regularly is the case. However, ultimately, it is an empirical
question whether this is the case.

While we started this chapter with an example of an OLS regression to
estimate a continous outcome (well, by approximation, as for ``days
per week'' not all values make sense), we will now use a regression
approach to predict nominal outcomes, just as in the Na\"ive Bayes
example.  The type of regression analysis to use for this is called
\emph{logistic regression}.

In a normal OLS regression, we estimate

$$y = \beta_o + \beta_1 x_1 + \beta_2 x_2 + \ldots + \beta_n x_n$$

But this gives us a continous outcome, which we do not want. In a
logistic regression, we therefore use the sigmoid function to map this
continous outcome to a value between 0 and 1. The sigmoid function is
defined as $sigmoid(z) = \frac{1}{1 + e^{-z}}$ and depicted in
Figure~\ref{fig:sigmoid}.

\begin{figure}
  \centering
\begin{tikzpicture}
    \begin{axis}%
    [
        grid=major,     
        xmin=-6,
        xmax=6,
        axis x line=bottom,
        ytick={0,.5,1},
        ymax=1,
        axis y line=middle,
    ]
        \addplot%
        [
            black,%
            mark=none,
            samples=100,
            domain=-6:6,
        ]
        (x,{1/(1+exp(-x))});
    \end{axis}
\end{tikzpicture}
\caption{\label{fig:sigmoid} The sigmoid function}
\end{figure}


Combining these formulas gives us:

$$P = \frac{1}{1 + e^{-(\beta_o + \beta_1 x_1 + \beta_2 x_2 = \ldots + \beta_n x_n)}} $$


Wait, you might say. Isn't $P$ still continuous, even though it is now
bounded between 0 and 1? Yes, it is.  Therefore, after having
estimated the model, we use a threshold value (typically, 0.5, but we will discuss in \refsec{balance} how to select different ones) to
predict the label. If $P>0.5$, we predict that the case is spam/about
politics/will go viral, if not, we predict it's not.

A nice side effect of this is that we still can use the probabilities
in case we are interested in them, for example to figure out for which
cases we are more sure in our prediction.

Just as with the Na\"ive Bayes classifier, also for logistic
regression classifiers, Python and R will happily allow us to estimate
models with multiple nominal outcomes instead of a binary outcome. In \refex{logreg} we fit the logistic regression using the \pkg{caret} method \fn{logreg} in R and the \pkg{sklearn} (module linear\_model) function \fn{LogisticRegression} in Python.

And, of course, you actually can do OLS regression (or more advanced
regression models) if you want to estimate a continuous outcome.


\pyrex[output=none, caption=A simple logistic regression classifier]{chapter09/logreg}



\subsection{Support Vector Machines}
Support Vector Machines (SVM) are another very popular and versatile
approach to supervised machine learning.  In fact, they are quite
similar to logistic regression, but try to optimize a different
function. In technical terms, SVM minimizes \emph{hinge loss} instead
of logistic loss.

What does that mean to us? When estimating logistic regressions, we
are interested in estimating probabilities, while when training a
SVM, we are interested in finding a plane (more
specifically, a hyperplane) that best separates the datapoints of the
two classes (e.g., spam vs non-spam messages) that we want to
distinguish.  This also means that a SVM does not give you
probabilities associated with your prediction, but just the label.
But usually, that's all that you want anyway.

Without going into mathematical detail here (for that, a good source
would be \citet{kelleher2015fundamentals}), we can say that finding the
widest separating margin that we can achieve constructing a plane in a
graphical space (SVM) versus optimizing a log-likelihood function
(logistic regression) results in a model that is less sensitive to
outliers, and tends to be more balanced.

There are a lot of graphical visualizations available, for example in
the notebooks supplementing \citep{vanderplas2016python}
(\url{https://jakevdp.github.io/PythonDataScienceHandbook/05.07-support-vector-machines.html}).
For now, it may suffice to imagine the two-dimensional case: we
construct a line that separates two groups of dots \emph{with the
  broadest possible margin}. The dots that the margin of this line
just touches are called the ``support vectors'', hence the name.

You could imagine that sometimes, we may want to be a bit lenient
about the margins. If we have thousands of data points, then maybe it
is okay if one or two of these datapoints are, in fact, within the
margin of the separating line (or hyperplane). We can control this with
a parameter called $C$: For very high values, this is not allowed, but
the lower the value, the ``softer'' the margin is.  In
Section~\ref{sec:crossvalidation}, we will show an approach to find
the optimal value.

A big advantage of SVMs is that they can be extended to non-linear
separable classes. Using a so-called kernel function or \emph{kernel trick}, we can transform
our data so that the dataset becomes linearly separable. Choices
include but are not limited to multinomial kernels, the radial basis
function (RBF), or Gaussian kernels. If we, for example, have a two
concentric rings of datapoints (like a donut), then we cannot find a
straight line separating them. But a RBF kernel can transfer them into
a linearly separable space. The aforementioned online visualizations
can be very instructive here.

\refex{svm} shows how we implement standard SVM to our data using the \pkg{caret} method \fn{svmLinear3} in R and the \pkg{sklearn} (module svm) function \fn{SVC} in Python. You can notice in the code that feature data is standardized or normalized (with $m = 0$ and ${std} = 1$) before model training in order to have all the features measured at the same scale, as required by SMV.

\pyrex[output=none, caption=A simple Support Vector Machine classifier]{chapter09/svm}



\subsection{Decision Trees and Random Forests}
\label{subsec:randomforest}
In the models we discussed so far, we essentially were modeling linear
relationships. If the value of a feature is twice as high, its
influence on the outcome will be twice as high as well.  Sure, we can
(and do, as in the case of the sigmoid function) apply some
transformations, but we have not really considered yet how we can
model situations in which, for instance, we care about whether the
value of a feature is above (or below) a specific threshold.  For
instance, if we have a set of social media messages and want to model
the medium where they most likely come from, then its length is very
important information. If it is longer than 280 characters (or,
historically, 140), then we can be \emph{very} sure it is not from
Twitter, even though the reverse is not necessarily true. But it does
not matter at all whether it is 290 or $10000$ characters long.

Entering this variable into a logistic regression, thus, would not be
a smart idea.  We could, of course, dichotomize it, but that would
only partly solve the problem.  In this example, we \emph{know} how to
dichotomize it based on our prior knowledge about the number of
characters in a tweet, but this does not necessarily need to be the
case; it might be something we need to estimate.

A step-wise decision, in which we first check one feature (the
length), before checking another feature, can be modeled as a decision
tree.  Figure~\ref{fig:decisiontree} depicts a (hypothetical) decision
tree with three \textit{leaves}.

\begin{figure}
  \centering
\begin{tikzpicture}[
    node/.style={%
      draw,
      rectangle,
    },
  ]

    \node [node] (A) {Number of characters};
    \path (A) ++(-135:30mm) node [node] (B) {FB-post};
    \path (A) ++(-45:30mm) node [node] (C) {Uses hashtag?};
    \path (C) ++(-135:30mm) node [node] (D) {Tweet};
    \path (C) ++(-45:30mm) node [node] (E) {FB-post};

    \draw (A) -- (B) node [left,pos=0.25] {$>280$}(A);
    \draw (A) -- (C) node [right,pos=0.25] {$<280$}(A);
    \draw (C) -- (D) node [left,pos=0.25] {yes}(A);
    \draw (C) -- (E) node [right,pos=0.25] {no}(A);
\end{tikzpicture}
  \caption{\label{fig:decisiontree}A simple decision tree}
\end{figure}

Faced with the challenge to predict whether a social media message is
a tweet or a Facebook post, we could predict 'Facebook post' if its
length is greater than 280 characters. If not, we check whether it
includes hashtags, and if so, we predict 'tweet', otherwise, 'Facebook
post'.

Of course, this simplistic model will be wrong at some times, because
not all tweets have hashtags, and some Facebook posts actually do
include hashtags.

While we constructed this hypothetical decision tree by hand, usually,
we are more interested in learning such non-linear relationships from
the data.  This means that we do not have to determine the cutoff
point ourselves, but also that we do not determine the order in which
we check multiple variables by hand.

Decision trees have two nice properties. First, they are very easy to
explain.  In fact, a figure like Figure~\ref{fig:decisiontree} is
understandable for non-experts, which can be important in scenarios
where for accountability reasons, the decision of a classifier must be
as transparent as possible.  Second, they allow us to approximate
almost all non-linear relationships (be it not necessarily very
accurately).

However, this comes at large costs.  Formulating a model as a series
of yes/no questions, as you can imagine, inherently uses a lot of
nuance. More importantly, in such a tree, you cannot ``move up''
again. In other words, if you make a wrong decision early on in the
tree (i.e., close to its root node), you cannot correct it any more.
This rigidity makes decision trees also prone to overfitting: they may
fit the training data very well, but may not generalize well enough to
slightly different (test) data.

Because of these drawbacks, decision trees are seldom used in
real-life classification tasks.  Instead, one uses an \empg{ensemble} model: so-called random forests.  Drawing random samples from the data, we estimate multiple
decision trees -- hence, a forest.  To arrive at a final prediction,
we then can let the trees ``vote'' on which label we should
predict. This procedure is called ``majority voting'', but there are
also other methods available. For example, \pkg{scikit-learn} in Python  by default uses a method called \fn{probabilistic prediction}, which takes into
account probability values instead of simple votes.

In \refex{randomforest} we create a random forest classifier with 100 trees using the \pkg{caret} method \fn{rf} in R and the \pkg{sklearn} (module ensamble) function \fn{RandomForestClassifier} in Python.

\pyrex[output=none, caption=A simple Random Forest classifier]{chapter09/randomforest}

Because random forests alleviate the problems of decision trees, but
keep the advantage of being able to model non-linear relationships,
they are frequently used when we expect such relationships (or have no
idea about how the relationship looks like).  Also, random forests may
be a good choice if you have very different types of features (some
nominal, some continuous, etc.) in your model. The same holds true if
you have a lot (really a lot) of features: Methods like SVM would require
constructing large matrices in memory, which random forests do not.
But if the relationships between your features and your labels are
actually (approximately) linear, then you are probably better off with
one of the other models we discussed.


\subsection{Neural Networks} \label{sec:neural}

A final machine learning algorithm that we will discuss here are neural networks.
Inspired by the neurons in human brains (and that of other animals),
neural networks consist of connections between neurons that are activated if the
total input is above a certain threshold.

\reffig{perceptron} shows the simplest type of neural network, sometimes called a \concept{perceptron}.
This neural network consists only of a series of input neurons (representing the features or independent variables)
which are directly connected with the output neuron or neurons (representing the output class(es)).
Each of the connections between neurons has a weight, which can be positive or negative.
For each output neuron, the weighted sum of inputs is calculated and a function is applied to determine the result.
An example output function is the sigmoid function which transforms the output to a value between zero and one,
in which case the resulting model is essentially a form of logistic regression.

\begin{figure}
  \centering
 \begin{neuralnetwork}[height=4]
        \newcommand{\x}[2]{$x_#2$}
        \newcommand{\y}[2]{$\hat{y}_#2$}
        \inputlayer[count=3, bias=true, title=Input\\layer, text=\x]
        \outputlayer[count=1, title=Output\\layer, text=\y] \linklayers
 \end{neuralnetwork}
\caption{Schematic representation of a typical classical machine learning model \label{fig:perceptron}}
\end{figure}

If we consider a neural network for sentiment analysis of tweets,
the input neurons could be the frequencies of words such as `great' or `terrible',
and we would assume that the weight of the first would be positive while the second would be negative.
Such a network cannot take combinations into account, however:
the result of `not great' will simply be the addition of the results of `not' and `great'.

To overcome this limitation, it is possible to add a \concept{hidden layer} of latent variables
between the input and output layer, such as shown in \reffig{hiddenlayers}.
This allows for combinations of neurons, with for example both `not' and `great' loading onto a hidden neuron,
which can then override the direct effect of `great'.
An algorithm called \concept{backpropagation} can be used to iteratively approximate the optimal values for the model.
This algorithm starts from a random state and optimizes the second layer while keeping the first constant,
then optimizing the first layer, and repeating until it converges. 

Although such hidden layers, which can easily contain thousands of neurons, are hard to interpret substantively,
they can substantially improve the performance of the model.
In fact, the Universal Approximation theorem states that every decision function can be approximated to infinite precision
with a single (but possibly very large) hidden layer \citet{goldberg17}.
Of course, since training data is always limited there is a practical limit to how deep or wide the network can be,
but this shows the big difference that a hidden layer can make in the range of regularities that can be `captured' in the model. 

\begin{figure}
  \centering
 \begin{neuralnetwork}[height=5]
        \newcommand{\x}[2]{$x_#2$}
        \newcommand{\y}[2]{$\hat{y}_#2$}
        \newcommand{\hfirst}[2]{\small $h^{(1)}_#2$}
        \newcommand{\hsecond}[2]{\small $h^{(2)}_#2$}
        \inputlayer[count=3, bias=true, title=Input\\layer, text=\x]
        \hiddenlayer[count=5, bias=false, title=Hidden\\layer 1, text=\hfirst] \linklayers
        \hiddenlayer[count=3, bias=false, title=Hidden\\layer 2, text=\hsecond] \linklayers
        \outputlayer[count=1, title=Output\\layer, text=\y] \linklayers
 \end{neuralnetwork}
\caption{A neural network \label{fig:hiddenlayers}}
\end{figure}

%Neural networks with and without hidden layers can be easily trained in either Python or R,
%as shown in \refex{neuralnetwork}.
%% TODO: insert example

