\chapter{Processing text}
\label{chap:protext}


\begin{abstract}{Abstract}
Many data sets that are relevant for social science consist of textual data, from political discussions and newspaper archives to open-ended survey questions and reviews. This chapter gives an introduction in dealing with textual data using base functions in Python and (mostly) the \pkg{stringr} package in R. 
\end{abstract}

\keywords{Text representation, text cleaning, regular expressions}

\begin{objectives}
\item Understand how text is represented in the computer
\item Be able to clean up and alter text
\item Understand and be able to use regular expressions 
\end{objectives}


\newpage
\begin{feature}
  \textbf{Packages used in this chapter}\\
  This chapter introduces the packages for handling textual data.
  For R, this is mainly the \pkg{stringr} package (included in \tidyverse).
  In Python, most functions are built-in, but will show how to use these functions in \pkg{pandas} and also introduce the \pkg{regex} alternative to built-in regular expressions.
You can install these packages with the code below if needed  (see \refsec{installing} for more details):

\doublecodex{chapter10/chapter10install}

\noindent After installing, you need to import (activate) the packages every session:

\doublecodex{chapter10/chapter10library}

\end{feature}

\section{Reading and cleaning text}
\label{sec:readtext}

When dealing with textual data, an important step is to normalize the data. Such preprocessing ensures that noise is removed, and reduces the amount of data to deal with. We cover typical steps such as lowercasing and removing punctuation, HTML tags and boilerplate.


\section{Regular expressions}
\label{sec:regular}

Regular expressions are a powerful language to find strings that conform to a given pattern. For instance, we can extract usernames or email-addresses from text, or normalize spelling variations and improve the cleaning methods covered in the previous section.



