\chapter{Processing text}
\label{chap:protext}

Many data sets that are relevant for social science consist of textual data, from political discussions and newspaper archives to open-ended survey questions and reviews. This chapter gives an introduction in dealing with textual data using packages such as quanteda (R) or NLTK (Python), and shows how to solve common problems encountered when reading and processing text. 

\section{Reading and cleaning text}
\label{sec:readtext}

When dealing with textual data, an important step is to normalize the data. Such preprocessing ensures that noise is removed, and reduces the amount of data to deal with. We cover typical steps such as lowercasing and removing punctuation, HTML tags and boilerplate.


\section{Regular expressions}
\label{sec:regular}

Regular expressions are a powerful language to find strings that conform to a given pattern. For instance, we can extract usernames or email-addresses from text, or normalize spelling variations and improve the cleaning methods covered in the previous section.

\section{Basic natural language processing}
\label{sec:nlp}


This section outlines how to go a step further and use natural language processing (NLP) to, for instance, parsing sentences, stemming or lemmatization. Using such part-of-speech tagging (POS), we can for example only retain nouns and adjectives, or any other element of interest. Similarly, we can use named entity recognition (NER) to extract names of persons, organizations, or locations. We will use spacy (Python and R).

