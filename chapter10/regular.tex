\section{Regular expressions}
\label{sec:regular}

A regular expression or \emph{regex} is a powerful language to locate strings that conform to a given pattern. For instance, we can extract usernames or email-addresses from text, or normalize spelling variations and improve the cleaning methods covered in the previous section. Specifically, regular expressions are a sequence of characters that we can use to design a pattern and then use this pattern to \emph{find} strings (identify or extract) and also \emph{replace} those strings by new ones. 

We will show how you can use regexes preprocess text. The good thing is that regular expression syntax are kind of similar in R and Python and once you learn how to write a pattern in one language it is easy to do the same in the other.  However they are not identical!\footnote{Even it is out of the scope of this book, you can trace the specific standards of each language and its different versions. You will come across with some standards of the Portable Operating System Interface (POSIX), such as the Basic Regular Expressions (BRE) or Extended Regular Expressions (ERE), or also with the Perl Compatible Regular Expressions (PCRE)}, and you might choose which standard you will use. In the case of the Python \fn{re} module regexes match operations similar to Perl, but in the case of R \fn{grep} function you will have set a parameter (\verb|perl = TRUE|) to work with Perl compatible regular expressions.	


Find a patter and replace 