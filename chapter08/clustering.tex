\section{Clustering and dimensionality reduction}
\label{sec:clustering}

So far, we have reviewed traditional statistical exploratory and
visualization techniques that any social scientist must manage. A more
computational next step in your EDA wirkflow is using machine learning
(ML) to let our computer ``learn'' about our data and in turn give us
more initial insights.  ML is a branch of artificial intelligence that
uses algorithms to interact with data and obtain some patterns or
rules that characterize that data. We normally distinguish between
supervised machine learning (SML) and unsupervised machine learning
(UML). In \refchap{introsml}, we will come back to this distinction.
For now, it may suffice to say that the main characteristic of
unsupervised methods is that we do not have any measuremeant available
for a dependent variable, label, or categorization, for which we want
to learn how they can be predicted. Instead, we want to identify
\emph{patterns} in the data without knowing in advance how these may
look like. In this, unsupervised machine learning is very much of a
inductive, bottom-up technique (see \refchap{introduction} and
\cite{Boumans2016}).

In this chapter, we will focus on UML as a means of finding groups and
latent dimensions in our data, which can also help to reduce our
number of variables. Specifically, we will use base R and Python's
\pkg{scikit-learn} to conduct $k$-means clustering, hierarchical
clustering and principal component analysis (PCA) as well as the
closely related singular value decomposition (SVD).

In data mining, we use clustering as a UML technique that aims to find
the relationship between a set of descriptive variables. By doing
cluster analysis we can identify underlying groups in our data that we
will call \textit{clusters}. Imagine we want to explore how European
countries can be grouped based on their average support to
refugees/migrants, age and educational level. We might create some
\textit{a priori} groups (such as Southern vs. Northern countries),
but cluster analysis would be a great method to let the data ``talk''
and then create the most appropriate groups for this specific case. As
in all UML, the groups will come unlabelled and the computational
analyst will be in charge of finding an appropriate and meaningful
label for each cluster for a better communication of results. 

\subsection{$k$-means clustering}

$k$-means is a very frequently used algorithm to perform cluster
analysis. Its main advantage is that, compared to the hierarchical
clustering methods we will discuss later, it is very fast and does not
consume much resources. This makes it especially useful for larger
datasets.

$k$-means cluster analysis is a method that takes any number of observations (cases) and group them into a given number of clusters based on the proximity of each observation to the mean of the formed cluster (centroid).  Mathematically, we measure this proximity as the distance of any given point to its cluster center, and can be expressed as

$$J = \sum_{j=1}^{k} \sum_{i=1}^{n} \big\| X_i^{(j)} - c_j \big\| ^2$$

where $ \big\| X_i^{(j)} - C_j \big\| ^2$ is the distance between the data point $X_i^{(j)}$ and the center of the cluster $c_j$.

Instead of taking the mean, some variations of this algorithm take the median (k-medians) or a representative observation, also called medoid (k-medoids or partitioning around medoids, PAM) as a way to optimize the initial method.

Because $k$-means clustering calculates \emph{distances} between cases,
these distances need to be meaningful -- which is only the cases if
the scales on which the variables are measured are comparable. If all
your variables are measured on the same (continous) scale with the
same endpoints, you may be fine. In most cases, you need to normalize
your data by transforming them into, for instance, z-scores, or a
scale from 0 to 1.

Hence, the first thing we do in our example, is to prepare a proper
dataset with only continuous variables, scaling the data (for
comparability) and avoiding missing values (drop or impute). In
\refex{elbow}, we will use the variables support to refugees
(\emph{support\_refugees\_n}), support to
migrants(\emph{support\_migrants\_n}), age (\emph{age}) and
educational level (number of years of education)
(\emph{educational\_n}) and will create a dataframe \verb+d3+ with the
mean of all theses variables for each \emph{country} (each observation
will be a country). $k$-means requires us to specify the number of
clusters, $k$, in advance. This is a tricky question, and (besides of
arbitrarily deciding $k$!), you essentially need to re-estimate your
model multiple times with different $k$s.

The simplest method to obtain the optimal number of clusters is to
estimating the variability within the groups for different runs. This
means that wWe must run $k$-means for different number of clusters
(i.e. 1 to 15 clusters) and then choose the number of clusters that
decreases the variability maintaining the highest number of
clusters. When you generate and plot a vector with the variability ,
or in more technical words, the within-cluster sum of squares (WSS)
obtained after each execution, it is very easy to identify the optimal
number: Just look at the bend (\textit{knee} or \textit{elbow}) and
you will find the point where it decreases the most and then get the
optimal number of clusters (3 clusters in our example).

\pyrex[output=py,format=png,caption=Getting the optimal number of clusters]{chapter08/elbow}

Now we now estimate our final model (\refex{kmeans}). We generate 25
initial random centroids (the algorithm will choose the one that
optimizes the cost). The default of this parameter is 1, but it is
recommended to set it with a higher number (i.e. 20 to 50) to guaranty
the maximum benefit of the method. The base R function \fn{kmeans} and
\pkg{scikit-learn} function \fn{KMeans} in Python will produce the
clustering. You can observe the mean (scaled) for each variable in
each cluster, as well as the corresponding cluster for each
observation.

\pyrex[output=py,caption=Using Kmeans to group countries based on the average support of refugees and migrants\, age and educational level]{chapter08/kmeans}

Using the function \fn{fviz\_cluster} of the library \pkg{factoextra} in R, or the \pkg{pyplot} function \fn{scatter} in Python, you can get a visualization of the clusters. In \refex{kmeans2} you can clearly identify that the clusters correspond to Nordic countries (more support to foreigners, more education and age), Central and Southern European countries (middle support, lower education and age), and Eastern European countries (less support, lower education and age)\footnote{We can re-run this cluster analysis using k-medoids or partitioning around medoids (PAM) and get similar results (the three medoids are: Slovakia, Belgic and Denmark), both in data and visualization. In R you must install the package \pkg{cluster} than contains the function \fn{pam}, and in Python the package \pkg{scikit-learn-extra} with the function \fn{Kmedoids}.} .

\pyrex[output=r,format=png,caption=Visualization of clusters]{chapter08/kmeans2}


\subsection{Hierarchical clustering}

Another method to conduct a cluster analysis is hierarchical clustering, which builds a hierarchy of clusters that we can visualize in a dendogram.  This algorithm has two versions: a bottom-up approach (observations begin in their own clusters), also called \textit{agglomerative}, and a top-down approach (all observations begin in one cluster), also called \textit{divisive}. We will follow the bottom-up approach in this chapter and when you will look at the dendogram you will realize how this strategy repeatedly combines the two \textit{nearest} clusters at the bottom into a larger one in the top. The distance between clusters is initially estimated for every pair observation points and then put every point in its own cluster in order to get the closest pair of points and iteratively compute the distance between each new cluster and the previous ones. This is the internal rule of the algorithm and we must choose a specific linkage method (complete, single, average or centroid or Ward's linkage). Ward's linkage is a good default choice: It inimizes the variance of the clusters being merged. In doing so, tends to produce roughly evenly sized cluster and is less sensitive to noise and outliers than some of the other methods. 

In \refex{hc} we will use the function \fn{hcut} of the package \pkg{factoextra} in R and \pkg{scikit-learn} function \fn{AgglomerativeClustering} in Python, to compute the hierarchical clustering. 

A big advantage of hierarchical clustering is that, once estimated,
you can freely choose the number of clusters in which to group your
cases without re-estimating the model. If you decide, for instance, to
use four instead of three clusters after all, then the cases in one of
your three clusters are divided into two subgroups. With $k$-means, in
contrast, a three-cluster solution can be completely different from a
four-cluster solution. However, this comes at a big cost: hierarchical
clustering requires a lot more computing resources and may therefore
not be feasible for large datasets.

\pyrex[output=py,caption=Using hierarchical clustering to group countries based on the average support of refugees and migrants\, age and educational level]{chapter08/hc}

We can then plot the dendogram  with base R function \fn{plot} and \pkg{scipy} (module \texttt{cluster.hierarchy}) function \fn{dendogram} in Python. The summary of the initial model suggest us 2 clusters (size=2) but by looking at the dendogram you can choose the number of clusters you want to work with by choosing a height (for example 4 to get 3 clusters). 

\pyrex[output=py,format=png,caption=Dendogram to visualize the hierarchical clusterin]{chapter08/dendogram}

If you re-run the hierarchical clustering for 3 clusters (\refex{hc3}) and visualize it (\refex{vishc3}) you will get a graph similar to the one produced by $k$-means.

\pyrex[output=py,caption=Re-run hierarchical clustering with 3 clusters]{chapter08/hc3}

\pyrex[output=r,format=png,caption=Re-run hierarchical clustering with 3 clusters]{chapter08/vishc3}


\subsection{Principal component analysis and singular value decomposition}

Cluster analyses are in principle used to group similar
cases. Sometimes, we want to group similar variables instead.  A
well-known method for this is principal component analysis (PCA). This
unsupervised method is useful to reduce the dimensionality of your
data by creating new uncorrelated variables or \textit{components}
that describe the original dataset. PCA uses linear transformations to
create principal components that are ordered by the level of explained
variance (the first component will catch the largest variance). We
will get as many principal components as number of variables we have
in the dataset, but when we look at the cumulative variance we can
easily select only few of these components to explain most of the
variance and thus work with a smaller and summarised dataframe that
might be more convenient for many tasks (i.e. those that require
avoiding multicollinearity or just need to be more computationally
efficient). By simplifying the complexity of our data we can have a
first understanding of how our variables are related and also of how
our observations might be grouped. All omponents have specific loads
for each original variable, which can tell you how the old variables
are represented in the new components. This statistical technique is
especially useful in EDA when working with high dimensional datasets
but it can be used in many other situations.

The mathematics behind PCA can be relatively easy to
understand. However, for the sake of simplicity, we will just say that
in order to obtain the principal components the algorithm firstly has
to compute the mean of each variable and then compute the covariance
matrix of the data. This matrix contains the covariance between the
elements of a vector and the output will be a square matrix with
identical number of rows and columns, corresponding to the total
number of dimensions of original dataset. Specifically, we can
calculate the covariance matrix of the variables \emph{X} and \emph{y}
with the next formula

$$cov_{x,y}=\frac{\sum_{i=1}^{N}(x_{i}-\bar{x})(y_{i}-\bar{y})}{N-1}$$

Secondly, using the covariance matrix the algorithm computes the eigenvectors and their corresponding eigenvalues, and then drop the eigenvectors with the lowest eigenvalues. With this reduced matrix it transforms the original values to the new subspace in order to obtain the principal components that will synthesize the original dataset.

Let us now conduct a PCA over the Eurobarometer data.  In \refex{pca} we will re-use the sub dataframe \emph{d3} containing the means of 4 variables (support to refugees, support to migrants, age and educational level) for each of the 30 European countries (30 x 4). The question is if we can have a new dataframe containing less than 4 variables but that explain most of the variance, or in other words, that represent well our original dataset with less dimensions. As long as our features are usually measured in different scales, it is normally suggested to center (to mean 0) and scale (to standard deviation 1) the data. We can perform the PCA in R using the base function \fn{prcomp} and in Python using the function \fn{PCA} of the module \texttt{decomposition} of \pkg{scikit-learn}. 

\pyrex[output=py,caption=Principal component analysis (PCA) of a dataframe with 30 records and 4 variables]{chapter08/pca}

The generated object with the PCA contains different elements (in R "sdev",     "rotation", "center",  "scale" and   "x") or attributes in Python (components\_, explained\_variance\_, explained\_variance\_ratio, singular\_values\_, mean\_, n\_components\_, n\_features\_, n\_samples\_ and noise\_variance\_). In the resulting object we can see the values of 4 principal components of each country, and the values of the loadings, technically called \textit{eigenvalues}, for the variables in each principal component. In our example we can notice that support to refugees and migrants are more represented on PC1, while age and educational level on PC2. If we plot the first two principal components using base function \fn{biplot} in R and and the library \pkg{bioinfokit} in Python (\refex{plot_pca}), we can clearly see how the variables are associated with either PC1 or with PC2 (we might also want to plot any pair of the four components!). But we can also get a picture of how countries are grouped based only in this two new variables.

\pyrex[output=py,format=png,caption=Plot PC1 and PC2]{chapter08/plot_pca}

So far we are not sure of how many components are enough to accurately represent our data, so we need to know how much variance (which is the square of the standard deviation) is explained by each component. We can get the values (\refex{prop}) and plot  the proportion of explained variance (\refex{prop2}). We get that the first component explain 57.85\% of the variance, the second 27.97\%, the third 10.34\% and the fourth just 3.83\%. 

\pyrex[output=py,caption=Proportion of variance explained]{chapter08/prop}
\pyrex[output=py,format=png,caption=Plot of the proportion of variance explained]{chapter08/prop2}

When we estimate (\refex{acum}) and plot (\refex{acum2}) the cumulative explained variance it is easy to identify that with just the two first components we explain 88.82\% of the variance. It might now seem a good deal to reduce our dataset from 4 to 2 variables, or let’s say half of the data, but retaining most of the original information.

\pyrex[output=py,caption=Cumulative explained variance]{chapter08/acum}
\pyrex[output=py,format=png,caption=Plot of the cumulative explained variance]{chapter08/acum2}

And what if we want to use this PCA and deploy a clustering (as explained above) with just these two new variables instead of the four original ones?  Just repeat the $k$-means procedure but now using a new smaller dataframe selecting PC1 and PC2 from the PCA. After estimating the optimal number of clusters (3 again!) we can compute and visualize the clusters, and get a very similar picture to the one obtained in the previous examples, with little differences such as the change of cluster of the Netherlands (more similar now to the Nordic countries!). This last exercise is a good example of how to combine different techniques in EDA.

\pyrex[output=none,caption=Combining PCA to reduce dimensionality and $k$-means to group countries]{chapter08/new}

When your dataset gets bigger, though, you may actually not use PCA
but the very much related singular value decomposition, SVD. They are
closely interrelated, and in fact SVD can be used ``under the hood''
to estimate a PCA. While PCA is taught in a lot of classical textbooks
for statistics in the social sciences, SVD is usually not. Yet, it has
a great advantage: In the way how it is implemented in
\pkg{scikit-learn}, it does not require to store the (dense)
covatiance matrix in memory (see the feature box on
p.~\pageref{feature:sparse} for more information on sparse versus dense
matrices). This means that while once your dataset grows bigger than
typical survey datasets, a PCA may be quickly impossible to estimate,
while the SVD can still be estimated without much ressources
needed. Therefore, especially when you are working with textual data,
you will see that SVD is used instead of PCA. For all practical
purposes, the way how you can use and interpret the results stays the
same, though.
