



\subsection{DO WE REALLY NEED THIS - CLASSES. IT'S A BIT TOO FAST AND TOO ADVANCED HERE I THINK. I THINK TO UNDERSTAND CLASSES, ONE NEEDS TO UNDERSTAND FUNCTIONS FIRST, AND WE NEED TO TALK ABOUT INHERITENCE, INSTANCES, ...}

Now that you have gone through most types of objects in R and Python, we need to remember that objects are different from classes, as we mentioned at the beginning of this section. A class is a prototype with attributes that helps to build an object when it is created from that class. R has three class systems: S3 class, S4 class and Reference Class.  Imagine that we have some data of a news item that we want to convert to a class using the basic S3 class:

\begin{exampler}
news <- list(headline = "Scientists discover how to cure AIDS", date = 21012030, positive_tone = TRUE)
class(news) <- "news_piece"
\end{exampler}

Or using Reference Class:

\begin{exampler}
setRefClass("news_piece")
\end{exampler}

Then you have stored the information of the class news\_piece.  You can do this also with S4 class, which allows you to define the formal structure of the class:

\begin{exampler}
setClass("news_piece", slots=list(headline="character", date="numeric", positive_tone="logical"))
\end{exampler}

Using S4 class definition you can create a new object using the attributes of the created class, which is very useful to maintain consistency:

\begin{exampler}
news2 <- new("news_piece", headline="Thousands of new refugees run from hunger after war begins", date=22012030, positive_tone = FALSE)
\end{exampler}

On the contrary, Python has just a single class. We define the class in a similar way we will define functions (see next section). We use the keyword class and, in some cases, the built-in \_\_init\_\_() function.  Let's create a very single class in Python including a numeric object in the class and later using that class to define a new object:

\begin{examplepy}
class MyWeight:
...    	 x = 74
weigth = MyWeight()
print (weigth.x)
\end{examplepy}

As you see, we created the object \emph{weight} using the properties (x in this case) of the class MyWeight. This procedure is probably of limited use but good to illustrate what a class is in Python. A more interesting approach to create the class is using the \_\_init\_\_() function, which is a special function that assigns values to object properties and is executed when the class is being initiated. In simpler words, this function helps us to assign different values within the class, creating an internal structure that will be used when we create an object based on that class. You can run the next example on your Python console for a better understanding of the concept. Imagine you want to create the same class news\_piece we did in R and then use that class to create an object:

\begin{examplepy}
class news_piece:
...     def __init__(self, headline, date, positive_tone):
...             self.headline = headline
...             self.date = date
...             self.positive_tone = positive_tone
news2 = news_piece("Thousands of new refugees run from hunger after war begins", 22012030, False)
print(news2.headline)
\end{examplepy}

As you see, with the last line of the code we obtain a part of the created object (in this case the headline) based on the structure designed in the class news\_piece.

In the next section you will learn the basics of how to write code in R and Python.
