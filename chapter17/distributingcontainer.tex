\section{Distributing your software as container}
\label{sec:container}
When publishing your software, you can think of multiple user
groups. Some maybe interesting in building on and further developing
your code. Some may not care about your code at all and just want your
software to run. And many others will be somewhere in between.

\emph{Only} publishing your source code (\refsec{publishingsource}) may
be a burden for those who want your code to ``just run'' once your
code becomes more complex and has more dependencies. Imagine a
scenario where your software requires a specific version of
Python or R and/or some very specific (or maybe incompatible) libraries
that you do not want to force the user to install.

And maybe your prospective user does not even know any R or Python.

For such cases, so-called containers are the solution, with as most
prominent platform \pkg{Docker}. You can envision a container as a
minimalistic virtual machine that includes everything to run your
software. To the outside, none of that is visible -- just a network
port to connect to, or a command line to interact with, depending on
your choices.

Software that is containerized using docker is distributed as a
so-called \emph{docker image}. You can build such an image yourself,
but it can also be distributed by pushing it to a so-called registry,
such as the \pkg{Docker Hub}. If you publish your software this
way, the end user has to do nothing else then installing Docker and
running the command |docker run nameofyourimage| - if will be even
downloaded automatically if necessary. There are also GUI versions
of Docker available, which lowers the threshold for the some end user
groups even more.

Let's illustrate the typical workflow with a toy example. Imagine
you wrote the following script, |myscript.py|:

\texttt{
import numpy as np

from random import randint

a = randint(0,10)

print(f"exp({a}) = {np.exp(a)}")
}

You think that this is an awesome program (after all, it calculates
$e$ to the power of a random integer!), and others should be able
to use it. And you don't want to bother them with setting up Python,
installing numpy, and then running the script. In fact, they do
not even need to \emph{know} that it's a Python program. You
could have written it as well in R, or any other langauge -- for
the user, that will make no difference at all.

What would a docker image that runs this code need to contain? Not
much: First some basic operating system (usually, a tiny Linux distribution),
Python, numpy, and the script itself.

To create such a Docker image, you create a file named |Dockerfile|
in the same directory as your script with the following content:

\texttt{
FROM python:3

ADD myscript.py /

RUN pip install numpy

CMD [ "python", "./myscript.py" ] }


The first line tells Docker to build your new image by starting
from an existing image that already contains an operating system
and Python3. You could also start from scratch here, but this
makes your life much easier. The next line adds your script to the
image, and then we run |pip install numpy| within the image.
The last line just specifies which command with which parameters
needs to be executed when the image is run -- in our case
|python ./myscript.py|.

To create the image, you run |docker build -t dockertest .| (if
you want to name the image ``dockertest''. After that, you can run
it using |docker run dockertest| -- and, if you want to, publish it.

Easy, right?

But when does it make sense to use Docker? Not in our toy example,
of course. While the original code is only a couple of bytes, it now
got bloated to hundreds of megabytes. But there are plenty of
scenario's where this makes a lot of sense.

\begin{itemize}
\item To ``abstract away'' the inner workings of your code. Rather than giving potentially complicated instructions how to run your code, which dependencies to install, etc., you can just provide users with the Docker image, in which everyting is already taken care of.
\item To ensure that users get the same results. Through it doesn't form a huge problem on a daily bases for most computational scientists, different versions of different libraries on different systems may ocassionally produce slightly different results. The container ensures that the code is run using the same software setup.
\item To avoid interfering with existing installations. Already our toy example had a dependency, numpy, but often, dependecies can be more complex and a program we write may need very specific libraries, or even some other software beyond Python or R libraries. Distributing the source code alone means forcing the user to also install these; and there are many good reasons why people may be reluctant to do so. It may be incompatible with other software on their computer, there may be security concerns, or it just may be too much work. But if it runs inside of the docker container, many of these problems disappear.
\end{itemize}

In short, the Docker image is rarely the \emph{only} way in which you distribute your source code. But already adding a Dockerfile to your github repository so that users can build a Docker container can offer another and maybe better way of running your software to your audience.
