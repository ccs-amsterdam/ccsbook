\section{Distributing your software as container}
\label{sec:container}
When publishing your software, you can think of multiple user
groups. Some maybe interesting in building on and further developing
your code. Some may not care about your code at all and just want your
software to run. And many others will be somewhere in between.

\emph{Only} publishing your source code \refsec{publishingsource} may
be a burden for those who want your code to ``just run'' once your
code becomes more complex and has more dependencies. Imagine a
scenario where your software requires a specific version of
Python or R and/or some very specific (or maybe incompatible) libraries
that you do not want to force the user to install.

And maybe your prospective user does not even know any R or Python.

For such cases, so-called containers are the solution, with as most
prominent platform \pkg{Docker}. You can envision a container as a
minimalistic virtual machine that includes everything to run your
software. To the outside, none of that is visible -- just a network
port to connect to, or a command line to interact with, depending on
your choices.

\todo[inline]{ADD EXAMPLE}
