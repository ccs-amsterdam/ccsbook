%%%%%%%%%%%%%%%%%%%%%%%%%%%%%%%%%%%%%%%%%%%%%%%%%%%%%%%%%%%%%%%%
%% Title Pages
%% Wiley will provide title and copyright page, but you can make
%% your own titlepages if you'd like anyway
%% Setting up title pages, type in the appropriate names here:

\booktitle{Computational Analysis of Communication}

\subtitle{A practical introduction to the analysis of texts, networks, and images with code examples in Python and R}

\AuAff{Wouter van Atteveldt\\Vrije Universiteit Amsterdam}
\AuAff{Damian Trilling\\University of Amsterdam}
\AuAff{Carlos Arcila Calder\'on\\University of Salamanca}

%% \\ will start a new line.
%% You may add \affil{} for affiliation, ie,
%\authors{Robert M. Groves\\
%\affil{Universitat de les Illes Balears}
%Floyd J. Fowler, Jr.\\
%\affil{University of New Mexico}
%}

%% Print Half Title and Title Page:
\halftitlepage
\titlepage

%%%%%%%%%%%%%%%%%%%%%%%%%%%%%%%%%%%%%%%%%%%%%%%%%%%%%%%%%%%%%%%%
%% Copyright Page

\begin{copyrightpage}{2020}
Title, etc
\end{copyrightpage}

% Note, you must use \ to start indented lines, ie,
% 
% \begin{copyrightpage}{2004}
% Survey Methodology / Robert M. Groves . . . [et al.].
% \       p. cm.---(Wiley series in survey methodology)
% \    ``Wiley-Interscience."
% \    Includes bibliographical references and index.
% \    ISBN 0-471-48348-6 (pbk.)
% \    1. Surveys---Methodology.  2. Social 
% \  sciences---Research---Statistical methods.  I. Groves, Robert M.  II. %
% Series.\\

% HA31.2.S873 2004
% 001.4'33---dc22                                             2004044064
% \end{copyrightpage}

%%%%%%%%%%%%%%%%%%%%%%%%%%%%%%%%%%%%%%%%%%%%%%%%%%%%%%%%%%%%%%%%
%% Only Dedication (optional) 

\dedication{To our patient spouses}

\tableofcontents

%\listoffigures %optional
%\listoftables  %optional

%% or Contributor Page for edited books
%% before \tableofcontents

%%%%%%%%%%%%%%%%%%%%%%%%%%%%%%%%%%%%%%%%%%%%%%%%%%%%%%%%%%%%%%%%
%  Contributors Page for Edited Book
%%%%%%%%%%%%%%%%%%%%%%%%%%%%%%%%%%%%%%%%%%%%%%%%%%%%%%%%%%%%%%%%

% If your book has chapters written by different authors,
% you'll need a Contributors page.

% Use \begin{contributors}...\end{contributors} and
% then enter each author with the \name{} command, followed
% by the affiliation information.

%\begin{contributors}
%\end{contributors}

%%%%%%%%%%%%%%%%%%%%%%%%%%%%%%%%%%%%%%%%%%%%%%%%%%%%%%%%%%%%%%%%
% Optional Foreword:

%\begin{foreword}
%\end{foreword}

%%%%%%%%%%%%%%%%%%%%%%%%%%%%%%%%%%%%%%%%%%%%%%%%%%%%%%%%%%%%%%%%
% Optional Preface:

%\begin{preface}
%\prefaceauthor{}
%\where{place\\date}
%\end{preface}

% ie,
\begin{preface}
Why write another methods textbook? Aren't there enough textbooks already? And what about all the great online ressources?
We have been teaching computational analysis of communication for years in different universities and organizations
for years, spanning different countries, different formats from semester-long courses to short workshops, different techniques,
and different levels -- but we never found the book that really fitted our audience. Regularly, students and colleagues ask
us for book recommendations; educators and admininstrators want to know which book we want to put on a reading list. And regularly,
our answer was one along the lines of: Well, there is this great book on [R/Python/Neural Networks/\ldots], but \ldots.
The ``but'', in almost all cases, has to do with the audience: students of the social sciences who have at least some knowledge of and are
interested in empirical research and quantitative methods, but have no experience in programming. They do want (or have to)
learn programming to conduct the analyses they are interested in, but are not necessarily interested in programming for its own sake.
They do not want to just push a button in some tool that limits their possibilities to what someone else has designed, but they
also do not want to follow a whole Introduction to Computer Science with a comprehensive overview of programming concepts and paradigms
that they might never need.

For years, we have therefore used our own materials to find a balance between teaching programming concepts where necessary but
focussing on their application for answering questions that are of genuine interest to those studying various forms of communication.
This book is our attempt to bring together and systematisize this approach to teaching the analysis of communication.

A second driver for writing this book was to get over the ``language war'' that is sometimes visible in the field. In our own research
and teaching, we find both R and Python to be great tools, with both their own strengths and weaknesses. Too often, existing
teaching materials focus on the language rather than the underlying concept. We believe that a good computational methods text book
should give practical instructions on the implementation of a concept in a given language, but put the concept rather than the language at the forefront.

This book would not have been possible without the continous input we got over years -- from students, colleagues, and others.
They shaped our ideas on both how to analyze communication computationally, but also our ideas about how to teach this. It
would also not have been possible without the patience of Nel, Rodrigo, and Sanne, when we again had to spend more hours than
we thought on what at one point only became known as ``the book''.

\prefaceauthor{Wouter van Atteveldt\\ Damian Trilling \\ Carlos Arcila Calder\'on}
\where{Amsterdam, Salamanca, Texel, \& online}
\end{preface}

%%%%%%%%%%%%%%%%%%%%%%%%%%%%%%%%%%%%%%%%%%%%%%%%%%%%%%%%%%%%%%%%
% Optional Acknowledgments:

\acknowledgments
Dmitry Bogdanov, Cecil Meeusen, Jesús Sánchez-Oro,

For an earlier version of the example for web scraping with Selenium, we would like to thank Marthe Möller.


%%%%%%%%%%%%%%%%%%%%%%%%%%%%%%%%
%% Glossary Type of Environment:

% \begin{glossary}
% \term{<term>}{<description>}
% \end{glossary}

%%%%%%%%%%%%%%%%%%%%%%%%%%%%%%%%
%\begin{acronyms}
%\end{acronyms}

%\begin{introduction}

%Maybe just read the friggin chapter?
%\end{introduction}
