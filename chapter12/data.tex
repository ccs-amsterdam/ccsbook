\section{Obtaining Review Data}

For the sections on dictionary and supervised approaches we will use a dataset of movie reviews
from the IMDB database \citet{aclimdb}.
This dataset is published as a compressed set of folders, with separate folders for the train and test datasets and subfolders for positive and negative reviews.
There are lots of other review data sets posted, for example for Amazon review data (\url{https://jmcauley.ucsd.edu/data/amazon/}). Another nice 

The IMDB data set we will use is a relatively large file and it requires bit of processing,
so it is smart to \emph{cache} the data rather than downloading and processing it every time you need it.
This is done in \refex{reviewdata}, which also serves a nice example of how to download and process files.
Both R and Python follow the same basic pattern.
First, we check whether the cached file exists, and if it does we read the data from that file.
For R, we use the standard \concept{RDS} format, while for python we use a compressed \concept{pickle} file.
The format of the data is also slightly different, following the convention for each language:
In R we use the data frame returned by \fn{readtext},
which can read files from a folder or zip archive and return a data frame containing one text per row.
In Python, we have separate lists for the train and test data sets and for the full texts and labels:
\ttt{X\_train} are the training texts and \ttt{y\_train} are the corresponding labels.

\begin{ccsexample}
  \doublecodex{chapter12/reviewdata}
  \codexoutputtable{chapter12/reviewdatatable.r}
  \caption{Downloading and caching IMDB review data}\label{ex:reviewdata}
\end{ccsexample}

If the cached data file does not exist yet,
the file is downloaded from the Internet.
In R, we then extract the file and call \fn{readtext} on the resulting folder.
This automatically creates columns for the subfolders, so in this case for the dataset and label.
After this, we remove the download file and the extracted folder,
clean up the \ttt{reviewdata}, and save it to the \ttt{reviewdata.rds} file.
In Python, we can extract files from the downloaded file directly,
so we do not need to explictly extract it.
We loop over all files in the archive, and use a regular expression to
select only text files and extract the label and data set name
(see \refsec{regular} for more information about regular expressions).
Then, we extract the text from the archive, and add the text and the label to the appropriate list.
Finally, the data is saved as a compressed pickle file,
so the next time we run this cell it does not need to download the file again. 

