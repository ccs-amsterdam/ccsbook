
%\keywords{Open Science, Reproducibility, Ethics}
%\item Understand the imporance of Transparent, Open, and Ethical Science
%\item Understand the particular challenges for Open Computational Science
%\item Understand the tradeoffs between Transparency and Privacy


\section{Open, transparent, and ethical computational science}
\label{sec:ethics}


We started this book by reflecting on what we are actually doing when conducting computational analyses of communication.
One of the things we highlighted in \refchap{introduction} was our use of open-source tools, in particular Python and R
and the wealth of open-source libraries that extend them. Hopefully, you have also realized not only how much your work could
therefore built on the work of others, but also how many of the ressources you used were created as a community effort.

Now that you aquired the knowledge it takes to conduct computational research on communication, it is time to reflect
on how to give back to the community, and how to contribute to an open research environment. At the same time, it is
not as simple as ``just putting everything online'' -- after all, researchers often work with sensitive data. We therefore
conclude this book by a short discussion on open, transparent, and ethical computational science.


\paragraph{Transparent and Open Science}
In the wake of the so-called reproducibility crisis, the call for transparent and open science has become louder and louder
in the last years. The public, funders, and journals increasingly ask for access to data and analysis script that underly
published research. Of course, publishing your data and code is not a panacea for all problems, but it is a step towards
better science from at least two perspectives \citep{VanAtteveldt2019}: First, it allows others to reproduce your work, enhancing its credibility
(and the credibility of the field as a whole). Second, it allows others to build on your work without reinventing the wheel.

So, how can you contribute to this? Most importantly, as we advised in \refsec{practices}: use a version control system
and share your code on a site like github.com. We also discussed code-sharing possibilities in \refsec{publishingsource}. Finally, you can find a template for organizing your code and data so that your
research is easy to reproduce at \url{https://github.com/ccs-amsterdam/compendium}. 

\paragraph{The privacy--transparency tradeoff} While the sharing of code is not particularly controversial, the sharing of
data sometimes is. In particular, you may deal with data that contain personally identifiable information. On the one
hand, you should share your data to make sure that your work can be reproduced -- on the other hand, it would be ethically
(and depending on your jurisdiction, potentially also legally) wrong to share personal data about individuals.
As \cite{boyd2012} write: ``Just because it is accessible does not make it ethical''. Hence, the
situation is not always black or white, and some techniques exist to find a balance between the two: you can remove
(or hash) information such as usernames, you can aggregate your data, you can add artificial noise. Ideally, you should
integrate legal, ethical, and technical considerations to make an informed decision on how to find a balance such that
transparency is maximized while privacy risks are minimized. More and more literature explores different possibilities \cite[e.g.][]{Breuer2020}.


\paragraph{Other ethical challenges in computational analyses} 
Lastly, there are also other ethical challenges that go beyond the use of privacy-sensitive data. Many tools we
use give us great power, and with that comes great responsibility. For instance, as we highlighted in
\refsec{ethicallegalpractical}, every time we scrape a website, we cause some costs somewhere. They may be
neglectable for a single http request, but they may add up. Similarily, out calculations on some cloud service
cause environmental costs. Before starting a large-scale project, we should therefore make a tradeoff between
the costs or damage we cause, and the (scientific) gain that we achieve.

In the end, though, we firmly believe that as computational scientists, we are well-equipped to contribute to the move towards
more ethical, open, and transparent science. Let's do it!
