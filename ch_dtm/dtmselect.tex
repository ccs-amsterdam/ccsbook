\section{Weighting and selecting documents and terms}
\label{sec:dtmselect}

So far, the DTMs you made in this chapter simply show the count of each word in each document.
Many words, however, are not informative for many questions.
This is especially apparent if you look at a \concept{word cloud},
essentially a plot of the most frequent words in a \concept{corpus}
(which is just a fancy word for a set of documents).

\pyrex[caption=Word Cloud of the US State of the Union corpus,format=png]{ch_dtm/wordcloud}

\refex{wordcloud} shows the word cloud for the state of the union DTM created above.
In R, this is done using the \quanteda\ function \fn{textplot\_wordcloud}.
In Python we need to work a little harder, since the only has the counts, not the actual words.
So, we sum the DTM columns to get the frequency of each word, and combine that with the feature names (words)
from the |CountVectorized| object |cv|. Then we can create the WordCloud and give it the frequencies to use.
Finally, we plot the cloud and remove the axes.

The results from Python and R look differently at first -- for one think, R is nice and round but Python has more colors!
However, if you look at the cloud you can see both are not very meaningful: the largest words are all punctuation or words like
`a', `and', or `the'.
You have to look closely to find words like `federal' or `security' that give a hint on what the texts were actually about.

The reason for this is that the `noise' caused by punctuation and very common \concept{stop words} such as `a' and `the' is drowning out the signal of meaningful words.
Before doing a substantive analysis of a corpus, we almost always have to perform a number of steps to clean and preprocess it.
The remainder of this chapter will discuss the most common cleaning and preprocessing steps,
show how they can be performed in R and Python, and discuss when these are useful or not. 

\subsection{Removing stopwords}

A first step in cleaning a DTM can be \concept{stop word removal}.
Words such as `a' and `the' are often called stop words, i.e.  that do not tell us much about the content.
Both \quanteda\ and \sklearn\ include built-in lists of stop words, making it very easy to remove the most common words.
\refex{stopwords} shows the result of specifying `english' stop words to be removed for both packages.

\pyrex[caption=Simple stop word removal,format=png]{ch_dtm/stopwords}

Note, however, that it might seem easy to list words like `a` and `and',
but as it turns out there is no single well-defined list of stop words,
and (as always) the best choice depends on your data and your research question.

Linguistically, stop words are generally function words or closed word classes such as determiner or pronoun,
with closed classes meaning that while you can coin new nouns, you can't simply invent a new determiner or preposition.
However, there are many different stop word lists around, which make different choices and are compatible with
different kinds of preprocessing.
The Python word cloud in \refex{stopwords} shows a nice example of the importance of matching stopwords with the used
tokenization: a central `word' in the cloud is the contraction |'s|.
We are using the NLTK tokenizer, which splits |'s| from the word it was attached to, but the \sklearn\ stop word list
does not include that term.
So, it is important to make sure that the words created by the tokenization match how words appear in the stop word list.

As an example of the substantive choices inherent in using a stop word lists,
consider the word `will'.
As an auxilliary verb, this is probably indeed a stop word: for most substantive questions, there is no difference
whether you will do something or simply do it.
However, `will' can also be a noun (a testament) and a name (e.g. Will Smith).
Simply dropping such words from the corpus can be problematic; see \refsec{nlp} for ways of telling nouns and verbs apart
for more fine-grained filtering.

Moreover, some research questions might actually be interested in certain stop words.
If you are interesting in references to the future or specific modalities,
the word might actually be a key indicator. 
Similarly, if you are studying self-expression on Internet forums, social identity theory, or populist rhetoric,
words like `I', `us' and `them' can actually be very informative.

For this reason, it is always a good idea to understand and inspect what stop word list you are using,
and use a different one or customize it as needed \citep[see also][]{nothman18}.
\refex{stopwords2} shows how you can inspect and customize stop word lists.
See the package documentation for the \pkg{stopwords} packages in both Python and R
for more details on which lists are available and what choices these lists make. 

\pyrex[caption=Inspecting and Customizing stop word lists]{ch_dtm/stopwords2}

%J. Nothman, H. Qin and R. Yurchak (2018). “Stop Word Lists in Free Open-source Software Packages”. In Proc. Workshop for NLP Open Source Software.

\subsection{Removing punctuation and noise}

Next to stop words, text often contains punctuation and other things that can be considered `noise' for most research questions.
For example, it could contain emoticons or emoji, twitter hashtags or at-mentions, or html tags or other annotations.

In both python and R, we can use regular expressions to select which words to keep or remove.
As explained above in \ref{sec:regex}, regular expressions are a powerful way to specify (sequences of) characters which are to be kept or removed.
You can use this, for example, to remove things like punctuation, emoji, or HTML tags.
This can be done either before or after tokenizing (splitting the text into words). 
If you only want to keep or remove certain words, it is often easiest to do after tokenization
using a regular expression to select the words to keep or remove.
If you want to remove parts of words (e.g. to remove the leading `#` in hashtags) if is easiest to do that before tokenization,
that is, as a preprocessing step before the tokenization.

Finally, note that \quanteda\ contains a number of built-in functions for common use cases such as removing
punctuation, numbers, and twitter-punctuation (at-mentions and hashtags).
You can do these steps yourself using regular expressions in a very flexible way, but as the final part of example \refex{dtmregex}
shows, it is a lot easier to just use these options on the \fn{dfm} function. 


\subsection{Trimming and weighting a DTM}




\subsection{Weighting a DTM}

