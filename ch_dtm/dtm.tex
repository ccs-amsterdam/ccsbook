\section{The bag of words and term-document matrix}
\label{sec:dtm}

Before you can analyse text using the computer, the text must be represented in a way that is understandable for the computer.

The Document-term matrix (DTM, also called the term-document matrix or TDM) is a common numerical representation of text.
This represents a corpus as a matrix or table, where each row represents a document, each column represents a term (word),
and the numbers in each cell show how often that word occurs in that document.

\pyrex[caption=Example document-term matrix,output=r]{ch_dtm/dtm}

As an example, \refex{dtm} shows a DTM made from two lines from the famous poem by Mary Angelou.
The resulting matrix has two rows, one for each line; and 11 columns, one for each unique term (word).
In the columns you see the document frequencies of each term: the word ``bird'' occurs once in each line,
but the word ``with'' occurs only in the first line (text1) and not in the second (text2).

In R, you can use the \fn{dfm} function from the \pkg{quanteda} package.
This function can take a vector or column of texts and transforms it directly into a DTM.
In Python, you achieve the same by creating an object of the \cls{TextVectorizer} class, which has a \fn{fit\_transform} function.

\begin{feature}
  \textbf{R: Why is it a document-feature matrix?}
The package \pkg{quanteda} uses the term document-feature matrix because the columns can also contain
other information instead of terms, such as word pairs, and these are collectively called \emph{features} of the text.

\noindent\textbf{Python: Why fit\_transform?}
In Python, you don't have a function that directly transforms text into a DTM.
Instead, you create an \emph{transformer} called a CountVectorizer,
which can then be used to 'vectorize' texts (turn it into a row of numbers)
by counting how often each word occurs.
This uses the \fn{fit\_transform} function which is offered by all \sklearn\ transformers.
It `fits' the model on the training data, which in this case means learning the vocabulary.
It can then be used to transform other data into a DTM with the exact same columns,
which is often required for algorithms. 
\end{feature}

As you can see already in this simple example, the document-term matrix discards quite a lot of information from text.
Specifically, it disregards the order or words in a text: `John fired Mary' and `Mary fired John' both result in the same DTM,
even though the meaning of the sentences is quite different.
For this reason, a DTM is often called a `bag of words', in the sense that all words in the document are simply put in a big bag
without looking at the sentences or context of these words. 

Thus, the DTM can be said to be a specific and `lossy' representation of the text, that turns out to be quite useful for certain tasks:
The frequent occurrence of words like ``employment'', ``great'', or ``I'' might well be good indicators that a text is about the economy,
is positive, or contains personal expressions respectively.
As we will see in the next chapter, the DTM representation can be used for many different text analyses, from dictionaries to supervised and unsupervised machine learning.

Sometimes, however, you need information that is encoded in the order of words.
For example, in analysing conflict coverage it might be quite important to know who attack whom, not just that attacking took place.
In the last section of this chapter we will look at some ways to create a richer matrix-representation by using word pairs.
Although it is beyond the scope of this book,
you can also use automatic syntactic analysis to take grammatical relations into account as well.
As is always the case with automatic analyses, it is important to understand what information the computer is looking at,
as the computer cannot find patterns in information that it doesn't have.


