\chapter{Network Data}
\label{chap:network}

\begin{abstract}{Abstract}
  
Especially social media data, but also other types of data can often be represented as networks. This chapter introduces \pkg{igraph} (R+Python) and \pkg{networkx} (Python) to showcase how to deal with such data, perform Social Network Analysis (SNA) and represent it visually. 
\end{abstract}

\keywords{Graphs, Social Network Analysis}

\begin{objectives}
\item Understand how can networks be represented and visualized
\item Conduct basic description of networks
\item Perform Social Network Analysis
\end{objectives}



\newpage
\begin{feature}
  \textbf{Packages used in this chapter}\\
  This chapter uses functions from the package \pkg{igraph} in R and the package \pkg{networkx} in Python.
  In Python we will also use the \pkg{python-louvain} packages which introduces the Louvain clustering functions in \pkg{community}.
  %You will create graphs and social networks by yourself, and then will be able to perform different types of analysis, from description to clustering.
  You can install these packages with the code below if needed  (see \refsec{installing} for more details):

\doublecodex{chapter14/chapter14install}

\noindent After installing, you need to import (activate) the packages every session:

\doublecodex{chapter14/chapter14library}
\end{feature}


\section{Representing and visualizing networks}
\label{sec:graph}

How can networks help us to understand and represent social problems? Hos can we use social media as a source for small and large-scale network analysis? This section presents a brief overview of graph structures (nodes and edges) and types (directed, weighted, induced, etc.), together with their formats in R and Python. We also include basic graph analysis and visual representation. 
 
A graph is an elegant way to represent a set of elements and their relationships. The element could be a neuron, a person, an organization, a street or even a message, and the relationship could be a synapse, a trip, a commercial agreement, a drive connection or a content transmission. This is a different way to represent, model and analyse the world: instead of having rows and columns as in a typical dataframe, in a graph we have \textit{nodes} (components) and \textit{edges} (relations). The mathematical representation of a graph G=(V,E) is possible when we set the nodes (also called vertices): \[{v_{1}, v_{2},… v_{n}}\] And the edges or pair of nodes: \[{(v_{1}, v_{2}), (v_{1}, v_{3}), (v_{2},v_{3}) … (v_{m}, v_{n}) \in E}\] As you may imagine, it is a very versatile pr	ocedure to represent many kind of problems that includes social, media or political issues. In fact, if go back to 1934 we can see how graph theory (originally established in the 18th century) was firstly applied to the representation of social interactions (\cite{moreno1934shall}) in order to measure the attraction and repulsion of individuals of a social group.

The network approach in social sciences has an enormous potential to model and predict \textit{social influence}. There is empirical evidence that we can successfully apply this framework to explain distinct phenomena such as political opinions, obesity and happiness, given the influence of our friends (or even of the friends of our friends) over our behaviour. (\cite{christakis2009connected}). The network created by this sophisticated structure of human and social connections is an ideal scenario to understand how close we are from each other in terms of degrees of separation (\cite{watts2004six}) in small (i.e. a school) and large-scale (i.e. global pandemic) social dynamics. Moreover, the network approach can help us to track the propagation either of a virus in epidemiology, or a fake news in political and social sciences, such in the work by 
\citet{vosoughi2018spread}. 

Even if graph theory and SNA were widely used during the last century, we can say that the appearance of computers and the Internet generated a important milestone that triggered their potentiality. Firstly, computers allowed to ease the computation of graph measures and visualize their general and communal structures. Secondly, the emergence of a big spectrum of social media network sites (i.e. Facebook, Twitter, Sina Weibo, Instagram, Linkedin, etc.) produced an unprecedented number of online social interactions, which was certainly an excellent arena to apply this framework. Thus, the use of social media as a source for network analysis has become one of the most exciting and promising areas in the field of computational social science.

Now, let us show you how to conduct basic representation and visualization of networks in R and Python. As we mentioned above, the structure of a graph is based on nodes and edges, which are the fundamental components of any network. Suppose that we want to model the social network of five American politicians in 2017 (Donald Trump, Bernie Sanders, Hillary Clinton, Barack Obama and John McCain), based on their \textit{imaginary} connections in Facebook (friendship) and Twitter (follow). Technically, the base of any graph is a list of edges that can written as pair of nodes that indicate 	the relationship.  For instance, the friendship in Facebook between two politicians would normally be expressed as two strings separated by comma (i.e. "Hillary Clinton", "Donald Trump"). In \refex{graph} we use libraries \pkg{igraph} (R)\footnote{You can use this library in Python with the adapted package \pkg{python-igraph}}  and \pkg{networkx} (Python) to create from scratch a simple graph with 5 nodes and 4 edges, using the above-mentioned structure of pair of nodes (notice that we only include the edges and the vertices are automatically generated).

\pyrex[output=both,caption=Creating a graph from scratch]{chapter14/graph}

In both cases we generated a graph object \texttt{g1} which contains the structure of the network and different attributes (such as \verb|number_of_nodes()| in 	\pkg{networkx}) and default functions. In the case of Python you can notice that the \pkg{networkx} object behaves very similar to a dictionary, which is a kind of object you are already familiar with.

We can add or drop node and edges to this initial graph, or even modify the names of the vertices. One of the most useful functions is the visualization of the network (\fn{plot} in \pkg{igraph} and \fn{draw} or \fn{draw\_networkx} in \pkg{networkx}). \refex{visgraph} shows a basic visualization of the imaginary network of friendships of five American politicians in Facebook.

\pyrex[output=both,format=png,caption=Visualization of a simple graph]{chapter14/visgraph}

Using network terminology, either nodes or edges can be \textit{adjacent} or not. In the figure we can say that nodes representing Donald Trump and John McCain are adjacent because they are connected by an edge that depicts its friendship. Moreover, the edges representing the friendships between John McCain and Donald Trump, and Hillary Clinton and Donald Trump, is also adjacent because they share one node (Donald Trump).

Now you you might be wondering: What if I want to do the same in Twitter? Can I represent the relationship between users in the very same way than Facebook? Well, when you model networks is extremely important that you have a clear definition of what you mean with nodes and edges, in order to maintain a coherence interpretation of the graph. In both, Facebook and Twitter, the nodes represent the users, but the edges might not be the same. In Facebook, an edge represents the friendship between two users and this link \textit{has no direction} (once an user accepts a friend request, both users becomes friends). In the case of Twitter, an edge could represent if one user is following another user, but not the other way around! This means that the edge \textit{has a direction} and you can establish it in the graph. When you give directions to the edges you are creating a \textit{directed graph}. In \refex{directed} the directions are declared with the order of the pair of nodes: the first position is for the "from" and the second for the "to". In \pkg{igraph} (R) we  set the argument \verb+directed+ of the function \fn{make\_graph} to \verb+TRUE+. In \pkg{networkx} (Python), you use the class \cls{DiGraph} instead of \cls{Graph} to create the object \texttt{g2}.

\pyrex[output=both,caption=Creating a directed graph]{chapter14/directed}

In the new graph the edges represent the action of following a user in Twitter. The first declared edge indicates that Hillary Clinton follows Donald Trump, but does not indicate the opposite. In order to provide the directed graph with more \textit{arrows} we included in \texttt{g2} two new edges (Barack Obama following Hillary Clinton, and Clinton following Bernie Sanders), so we can have a couple of reciprocal relationships besides the unidirectional ones. You can visualize the directed graph in \refex{visdirected} and see how the edges now contains useful arrows.

\pyrex[output=both,format=png,caption=Visualization of a directed graph]{chapter14/visdirected}

The edges and nodes of our graph can also have some weights, features or attributes. When the edges have specific values that depict a feature of every pair of nodes (i.e. the distance between two cities) we say that we have a \textit{weighted graph}. This type of graphs are extremely useful to have a more accurate representation of a network. For example, in our hypothetical network of American politicians in Twitter (\texttt{g2}) we can weight the edges by including the number of likes that each politician has given to the followed user. This value  can serve as a measure of the distance between the nodes (i.e. the higher the number of likes the shorter the social distance). In \refex{weighted} we include the weights for each edge: Clinton has given 5 likes to Trumps' tweets, Sander 20 to Clinton's messages, and so on. In the plot you can see how the sizes of the lines between the nodes change as a function of the weights.

\pyrex[output=both,format=png,caption=Visualization of a weighted graph]{chapter14/weighted}

You can include more properties to the components of your graph. Imagine you want to use the number of followers of each politician to determine the size of the nodes, or the gender of the user to establish a color. In \refex{weighted2} we added the variable \emph{followers} to each of the nodes and asked the packages to plot the network using this value as size parameter (in fact we multiplied the values by 0.001 to make realistic on the screen, boy you could also normalize these values when needed). We also included the variable \emph{party} that was later recoded in a new one called \emph{color} in order represent Republicans with red and Democrats with blue.  You may need to add other features to the nodes or edges, but with this example you have an overview of what you can do.

\pyrex[output=both,format=png,caption=Visualization of a weighted graph]{chapter14/weighted2}

We can  mention a third type of graphs: the \textit{induced subgraphs}, which are in fact subsets of nodes and edges of a bigger graph. We can represent these subsets as G' = V', E'. In \refex{subgraph} we extract two induced subgraphs from out original network of American politicians in Facebook (\texttt{g1}): the first (\texttt{g3}) is built with the edges that contain nodes only Democrat nodes, and the second \texttt{g4} with edges formed by Republican nodes. There is also special case of induced subgraphs, called \textit{clique}, which is an independent or complete subset of an undirected graph (each node of the clique must have a way to connect to the rest of the nodes of the subgraph).

\pyrex[output=both,caption=Induced subgraphs for Democrats and Republicans]{chapter14/subgraph}

Keep in mind that in network visualization you can always configure the size, shape and color of your nodes or edges. It is out of the scope of this book to go into more technical details, but you can always check the online documentation of the recommended libraries.

So far we have created networks from scratch, but most of the time you will have to create a graph from an existing data file. This means that you will need an input data file with the graph structure, and some functions to load them as objects onto your workspace in R or Python. You can import graph data from different specific formats (i.e. Graph Modeling Language -GML-, GraphML, JSON, etc.), but one popular and standardized procedure is to obtain the data from a text file containing a list of edges or a matrix. In \refex{read}  we illustrate how to read graph data in \pkg{igraph} and \pkg{networkx} using a simple adjacency list that corresponds to our original imaginary Twitter network of American politicians \texttt{g2}.

\pyrex[output=both,caption=Reading a graph from a file]{chapter14/read}

\section{Social network analysis}
\label{sec:sna}

This section gives an overview of the existing measures to conduct Social Network Analysis (SNA). Among other functions, we explain how to examine paths and reachability, how to calculate centrality measures (degree, closeness, betweenness, eigenvector) to quantify the importance of a node in a graph, and how to detect communities in the graph using clustering.

\subsection{Paths and reachability}

The first idea that comes into our minds when analysing a graph is to understand how their nodes are connected. When multiple edges create a network we can observe how the vertices constitute one or many paths that can be described. In this sense, a \texttt{sequence} between node \textit{x} and node \textit{y} is a path where each node is \texttt{adjacent} to the previous. In the imaginary social network of friendship of American politicians contained in the undirected graph \texttt{g1}, we can determine the sequences or simple paths between any pair of politicians. As shown in \refex{path} we can use the function \fn{all\_simple\_paths} contained in both \pkg{igraph} (R) and \pkg{networkx} (Python), to obtain the two possible routes between Barack Obama and John McCain. The shortest path includes the nodes Hillary Clinton and Donald Trump; and the longer includes Bernie Sanders, Clinton and Trump.

\pyrex[output=both,caption=Possible paths between two nodes in the imaginary Facebook network of American politicians]{chapter14/path}

One specific type of path is the one in which the initial node is the same than the final node. This closed path is called a \texttt{circuit}. To understand this concept let us recover the inducted subgraph of Democrat politicians (\texttt{g3}) in which we only have three nodes. If you plot this graph, as we do in \refex{circuit}, you can clearly visualize how a circuit works. 

\begin{ccsexample}
  \doublecodex{chapter14/circuit}
  \codexpng{A circuit of three nodes}{.5\linewidth}{chapter14/circuit.r}
  \caption{Visualization of a circuit}
  \label{ex:circuit}
\end{ccsexample}

In SNA it is extremely important to be able to describe the possible paths since they help us to estimate the reachability of the vertices. For instance, if we go back to our original graph of American politicians on Facebook (\texttt{g1}) visualized in \refex{visgraph}, we can see that Sanders is reachable from McCain because there is a path between them (McCain - Trump - Clinton - Sanders).  Moreover, we observe that this social network is fully \texttt{connected} because you can reach any given node from any other node in the graph. But it might not always be that way. Imagine that we remove the friendship of Clinton and Trump by deleting that specific edge. As you can observe in \refex{component}, when we create and visualize the graph \texttt{g6} without this edge we can see that the network is not longer fully connected and it has two \texttt{components}. Technically speaking, we would say for example that the subgraph of Republicans is a connected component of the network of American politicians, given that this connected subgraph is part of the bigger graph while not connected to it.

\begin{ccsexample}
  \doublecodex{chapter14/component}
  \codexpng{Connected components}{.5\linewidth}{chapter14/component.r}
  \caption{Visualization of connected components}
  \label{ex:component}
\end{ccsexample}

When analysing paths and reachability you may be interested in knowing the distances in your graph. One common question is what is the average path length of a social network, or in other words, what is the average of the shortest distance between each pair of vertices in the graph. This \texttt{mean distance} can tell you a lot about how close the nodes in the network are: the shorter the distance the closer the nodes are. Moreover, you can estimate the specific distance (shortest path) between two specific nodes.  As shown in \refex{distance} we can estimate the average path length (1.7) in our imaginary Facebook network of American politicians using the functions \fn{mean\_distance} in \pkg{igraph} and \fn{average\_shortest\_path\_length} in \pkg{networkx}. In this example we also estimate the specific distance in the network between Obama and McCain (3) using the function \fn{distances} in \pkg{igraph} and estimating the length (\fn{len}) of the shortest path (first result of \fn{shortest\_simple\_paths} minus 1) in \pkg{networkx}.

\pyrex[output=both,caption=Estimating distances in the network]{chapter14/distance}

In terms of distance, we can also wonder what are the edges or nodes that share a border with any given vertex. In the first case, we can identify the \texttt{incident edges} that go out or into one vertex. As shown in \refex{incident}, by using the the functions \fn{incident} in \pkg{igraph} and \fn{edges} in \pkg{networkx} we can easily get incident edges of John McCain in Facebook Network (\texttt{g1}), which is just one single edge that joins Trump with McCain. In the second case, we can also identify its adjacent nodes, or in other words its \texttt{neighbours}. In the very same example, we use \fn{neighbors} (same function in \pkg{igraph} and \pkg{networkx}) to obtain all the nodes one step away from McCain (in this case only Trump).

\pyrex[output=both,caption=Incident edges and neighbours of J. McCain the imaginary Facebook Network]{chapter14/incident}

There are some other interesting descriptors of social networks. One of the most common measures is the \texttt{density} of the graph, which accounts for the proportion of edges relative to all possible ties in the network. In simpler words, the density tells us from 0 to 1 how much connected the nodes of a graph are. This can be estimated for both, undirected and directed graphs. Using the functions \fn{edge\_density} in \pkg{igraph} and \fn{density} in \pkg{networkx} we obtain a density of 0.5 (middle level) in the imaginary Facebook network of American politicians (undirected graph) and 0.35 in the Twitter network (directed graph). 

In undirected graphs we can also measure \texttt{transitivity} (also known as \texttt{clustering coefficient} and \texttt{diameter}. The first is a key property of social networks that refers to the ratio of triangles over the total amount of connected triples. It is to say that we wonder how likely is that two nodes are connected if they share a mutual neighbour. Applying the function \fn{transitivity} (included in \pkg{igraph} and \pkg{networkx}) to \texttt{g1} we can see that this tendency is of 0.5 in the Facebook network (there is 50\% of probability that two politicians 
are friends when they have a common contact). The second descriptor, the diameter, depicts the length of the network in terms of the longest geodesic distance\footnote{The \textit{geodesic distance} is the shortest number of edges between two vertices} . We use the function \fn{diameter} (included in \pkg{igraph} and \pkg{networkx}) in the Facebook network and get a diameter of 3, which you can also check if you go back to the visualization of \texttt{g1} in \refex{visgraph}. 

Additionally, in directed graphs we can calculate the \texttt{reciprocity}, which is just the proportion of reciprocal ties in a social network and can be computed with the function \fn{diameter} (included in \pkg{igraph} and \pkg{networkx}). For the imaginary Twitter network (directed graph) we get a reciprocity of 0.57 (which is not bad for a Twitter graph where important people usually have much more followers than follows!).

In \refex{density} we show how to estimate these four measures in R and Python. Notice that in some of the network descriptors you have to decide whether to include or not the edges weights for computation (in the provided examples we did not take these weights into account).

\pyrex[output=both,caption={Estimations of density, transitivity, diameter and reciprocity}]{chapter14/density}

\subsection{Centrality measures}

Now let us move to \texttt{centrality measures}. Centrality is probably the most common popular or known measure in the analysis of social networks because it gives you a clear idea of the importance of any of the nodes within a graph. Using its measures you can pose many questions such as which is the most central person in a network of friends in Facebook, who can be considered an opinion leader in Twitter or who is an influencer in Instagram. Moreover, knowing the specific importance of every node of the network can help us to visualize or label only certain vertices that overpass a previously determined threshold, or to use the color or size to distinguish the most central nodes from the others. There are four typical centrality measures: \texttt{degree}, \texttt{closeness}, \texttt{eigenvector} and \texttt{betweenness}.

The \texttt{degree} of a node refers to the number of ties of that vertex, or in other words, to the number of edges that are incident to that node. This definition is constant for undirected graphs in which the directions of the links are not declared. In the case of directed graphs, you will have three options to measure the degree. First, you can think of the amount of edges pointing \texttt{in} any node, which we call \texttt{in degree}; second, we have the number of edges pointing \texttt{out} any node, or \texttt{out degree}. In addition, we could also have the total number of edges pointing (in and out) any node. \texttt{Degree}, as well of other measures of centrality mentioned below, can be expressed in absolute numbers, but we can also \texttt{normalize}\footnote{The approach is to divide by the maximum possible number of vertices (N) minus 1, or by N-1. We may also estimate the \texttt{weighted degree} of a node, which is the same degree but ponderated by the weight of the edges.}  these measures for better interpretation and and comparison. We will prefer this latter approach in our examples, which is also the default option in many SNA packages.

We can then estimate the degree of two of our example networks. In \refex{centrality1} we first estimated the degree of each of the five American politicians in the imaginary Facebook network, which is an undirected graph; and then the total degree in the Twitter network, which is a directed graph. For both cases, we use the functions \fn{degree} in \pkg{igraph} (R) and \fn{degree\_centrality} in \pkg{networkx} (Python). We later compute the \texttt{in} and \texttt{out} degree for the Twitter network. In \pkg{igraph} we used again the function \fn{degree} but adjusting the parameter \texttt{mode} to "in" or "our", respectively; and in \pkg{networkx} we employed the functions \fn{in\_degree\_centrality} and \fn{out\_degree\_centrality}.

\pyrex[output=both,caption={Computing degree centralities in undirected and directed graphs}]{chapter14/centrality1}

There are three other types of centrality measures. \texttt{Closeness} centrality refers to the geodesic distance of a node to the rest of nodes in the graph. Specifically, it indicates how close a node is from the others by taking the length of the shortest paths between the vertices. \texttt{Eigenvector} centrality takes into account the importance of the surrounding nodes and computes the centrality of a vertex based on the centrality of its neighbours. In technical words, the measure is proportional to the sum of connection centralities. Finally, \texttt{betweenness} centrality indicates to which extent the node is in the paths that connect many other nodes. Mathematically it is computed as the sum of the fraction of every pair of (shortest) paths that go through the analysed node. 

As shown in \refex{centrality2}, we can obtain these three measures from undirected graphs using the functions \fn{closeness}, \fn{eigen\_centrality} and \fn{betweenness} in \pkg{igraph}, and \fn{closeness\_centrality}, \fn{eigenvector\_centrality} and \fn{betweenness\_centrality} in \pkg{networkx}. If we take a look to the centrality measures for every politician of the imaginary Facebook network we see that Clinton seems to be a very important and central node of the graph, just coinciding with the above-mentioned findings based on the degree. It is not a rule that we obtain the very same trend in each of the centrality measures but it is likely that they have similar results although they are looking for different dimensions of the same construct.

\pyrex[output=py,caption={Estimations of density, transitivity, diameter and reciprocity}]{chapter14/centrality2}

We can use these centrality measures in many ways. For example, you can take the degree centrality as a parameter of the node size and labelling when plotting the network. This may be of great utility since the reader can visually identify the most important nodes of the network while minimizing the visual impact of those that are less central. In \refex{plotsize} we decided to specify the size of the nodes (parameters \verb|vertex.size| in \pkg{igraph} and \verb|node_size| in \pkg{networkx}) with the degree centrality of each of the American politicians in the Twitter network (directed graph) contained in \texttt{g2}. We also used the degree centrality to filter the labels in the graph, and then included only those that overpassed a threshold of 0.5 (parameters \verb|vertex.label| in \pkg{igraph} and \verb|labels| in \pkg{networkx}). These two simple parameters of the plot give you a fair image of the potential of the centrality measures to describe and understand your social network.

\begin{ccsexample}
  \doublecodex{chapter14/plotsize}
  \codexpng{Network Plot}{.5\linewidth}{chapter14/plotsize.r}
  \caption{Using the degree centrality to change the size and labels of the nodes}
  \label{ex:plotsize}
\end{ccsexample}

\subsection{Clustering and community detection}

One of the greatest potentials of SNA is the ability to identify how nodes are interconnected and thus define \texttt{communities} within a graph. This is to say that most of the times the nodes and edges in our network are not distributed homogeneously, but they tend to form clusters that can  later be interpreted. In a social network you can think for example of the principle of \texttt{homophily}, which is the tendency of human beings to associate and interact with similar individuals; or you can think of extrinsic factors (i.e. economical or legal) that may generate the cohesion of small groups of citizens that belong to a wider social structure. With independence to the original cause, we use different computational approaches to model and detect possible communities that emerge from social networks and even to interpret and label those groups. The creation of clusters as an unsupervised machine learning technique was introduced in Section~\ref{sec:clustering} for structured data and in Section~\ref{sec:unsupervised} for text analysis (topic modeling). We will use some similar unsupervised approaches for community detection in social networks.

Many social and communication questions may arise when clustering a network. The identification of subgroups can tell us how diverse and fragmented a network is, or how the behaviour of a specific community relates to other groups and to the entire graph. Moreover, the concentration of edges in some nodes of the graph would let us know about the social structure of the networks which in turn would mean a better understanding of its inner dynamic.  It is true that the computational analyst will need more than the provided algorithms when labelling the groups to understand the communities, which means that you must get familiar with the way the graph has been built and what their nodes, edges or weights represent.

A first step to get involved with subgroups within a network is to find the available complete subgraphs in an undirected graph. As we briefly explained at the end of Section~\ref{sec:graph}, these independent subgraphs are called \texttt{cliques} and refer to subgroups where every vertex is connected to every other vertex. We can find the \texttt{maximal cliques} (a clique is maximal when it cannot be extended to a  bigger clique) in the imaginary undirected graph of American politicians in Facebook (\texttt{g1}) by using the functions \fn{max\_cliques} in \pkg{igraph} \citep{eppstein2010listing} and \fn{max\_cliques} in \pkg{networkx} \citep{cazals2008note}. As you can see in \refex{cliques}, we obtain a total of 3 subgraphs, one representing the Democrats, another the Republicans, and one more the connector of the two parties (Clinton-Trump).

\pyrex[output=both,caption={Finding all the maximal cliques in an undirected graph}]{chapter14/cliques}

Now, in order to properly detect communities we will apply some common algorithms to obtain the most likely subgroups in a social network. The first of these models is the so called \texttt{edge-between} or Girvan-Newman algorithm \citep{newman2004finding}. This algorithm is based on divisive hierarchical clustering (explained in Section~\ref{sec:unsupervised}) by breaking down the graph into pieces and iteratively removing edges from the original one. Specifically,  the Girvan-Newman approach uses the betweenness centrality measure to remove the most central edge at each iteration. You can easily visualize this splitting process in a dendrogram, as we do in \refex{girvan}, where we estimated \texttt{cluster1} to detect possible communities in the Facebook network. We used the functions \fn{cluster\_edge\_betweenness} in \pkg{igraph} and \fn{girvan\_newman} in \pkg{networkx}.

\begin{ccsexample}
  \doublecodex{chapter14/girvan}
  \codexpng{Dendogram}{.5\linewidth}{chapter14/girvan.r}
  \caption{Dendrogram to visualize clustering with Girvan-Newman}
  \label{ex:girvan}
\end{ccsexample}



When you look at the figure you will notice that the final leaves correspond to the nodes (the politicians) and then you have different partition levels (1 to 4), which in fact are different cluster possibilities. In edge-betweenness clustering, the big question is which partition level to choose, or in other words, which of the community division is better. The concept of \texttt{modularity} arises as a good measure (-1 to 1) to evaluate how good the division is (technically it's measured as the fraction of edges that fall within any given groups, let's say group 1 and group 2, minus the expected number of edges within those groups distributed at random). Thus, we can choose which of the four proposed divisions is the best based on the highest value of their modularities: The higher the modularity the more dense the connections \textit{within} the community and the more sparse the connections \textit{across} communities. In the case of \fn{cluster\_edge\_betweenness} in \pkg{igraph} it automatically estimates that the best division (on modularity) is the first one with two communities.

With community detection algorithms we can then estimate the length (number of suggested clusters), membership (to which cluster belongs each node) and modularity (how good is the clustering). In the case of \pkg{igraph} in R we apply the functions \fn{length} (base), \fn{membership} and \fn{modularity} over the produced clustering object (i.e. \texttt{cluster1}). In the case of \pkg{networkx} in Python we first have to specify that we want to use the first component of the divisions (out of 4) using the function \fn{next}. Then, we can apply the functions \fn{len} (base) and \fn{modularity} to get the descriptors, and print the fist division (stored as \texttt{communities1}) to obtain the membership (see \refex{girvan2}).

\pyrex[output=py,caption={Community detection with Girvan-Newman}]{chapter14/girvan2}

We can estimate the communities for our network using many other more clustering algorithms, such as \texttt{Louvain}, \texttt{Propagating Label} and \texttt{Greedy Optimization}, among others. Similar to Girvan-Newman, Louvain algorithm uses the measure of modularity to obtain a multi-level optimization \citep{blondel2008fast} and its goal is to obtain optimized clusters where the less number of edges between the communities and a higher number edges within the same community. For its part, the Greedy Optimization algorithm is also based on the modularity indicator \citep{clauset2004finding}. It does not consider the edges weights and works by initially setting each vertex in its own community and then joining two communities to increase modularity until obtaining the maximum modularity. Finally, the Propagating Label algorithm -which takes into account edges weights- initializes each node with a unique label and then iteratively each vertex adopts the label of its neighbours until all nodes have the most common label of their neighbours \citep{raghavan2007near}. The process can be conducted asynchronous (ass done in our example), synchronous or semi-synchronously (it might produce different results).

In \refex{clustalgo} we use \fn{cluster\_louvain}, \fn{cluster\_fast\_greedy} and \fn{cluster\_label\_prop} in \pkg{igrapgh} (R) and \fn{best\_partition}, \fn{greedy\_modularity\_communities} and \fn{asyn\_lpa\_communities} in \pkg{networkx} (Python). You can notice  that the results are quite similar\footnote{This similarity is because our example network is extremely small. In larger networks, the results might not be that similar.}  and it is pretty clear that there are two communities in the Facebook network: Democrats and Republicans!

\pyrex[output=py,caption={Community detection with Louvain\, Propagating Label and Greedy Optimization}]{chapter14/clustalgo}

We can plot each of those clusters for better visualization of the communities. In \refex{plotcluster} we generate the plots with the Greedy Optimization algorithm in R and the Louvain algorithm in Python, and we get two identical results.

\pyrex[output=both,format=png,caption=Plotting clusters with Greedy optimization in R and Louvain in Python]{chapter14/plotcluster}

There are more ways to obtain subgraphs of your network (such as the \texttt{K-core decomposition}) or to evaluate the homophily of your graph (using the indicator of \texttt{assortativity} that measures the degree to which the nodes associate to similar vertices). In fact, the are many other measures and techniques to conduct SNA that we have deliberately omitted in this section for reasons of space, but we have covered the most important aspects and procedures you need to know to initiate yourself in the computational analysis of networks.

So far we have seen how to conduct SNA over "artificial" graphs for the sake of simplicity. However, the representation and analysis of "real world" networks will normally be more challenging because of their size or their complexity. To conclude this chapter we will show you how to apply some of the explained concepts to real data.

Using the Twitter API (see \refsec{apis}), we retrieved the names of the first 100 followers of the five most important politicians in Spain by 2017 (Mariano Rajoy, Pedro Sánchez, Albert Rivera, Alberto Garzón and Pablo Iglesias). With this information we produced an undirected graph\footnote{We deliberately omitted the directions of the edges given its impossible reciprocity.} of the "friends" of these Spanish politicians in order to understand how these leader where connected through their followers. In \refex{friends1} we load the data into a graph object \texttt{g_friends} that contains the 500 edges of the network. As we may imagine the 5 mentioned politicians were normally the most central nodes, but if we look at the degree, betweenness and closeness centralities we can easily get some of the relevant nodes of the Twitter network: CEARefugio, elenballesteros or Unidadpopular. These accounts deserve special attention since they contribute to the connection of the main leaders of that country. In fact, if we conduct clustering analysis using Louvain algorithm we will find a high modularity (0.77, which indicates that the clusters are well separate) and not surprisingly 5 clusters. 

\pyrex[output=py,caption={Loading and analysing a real network of Spanish politicians and their followers in Twitter}]{chapter14/friends1}

When we visualize the clusters in the network (\refex{friends2}) using the degree centrality for the size of the node, we can locate the five politicians in the center of of the clusters (depicted with different colors). More interesting, we can see that even when some users follow two of the political leaders, they are just assigned to one of the cluster. This the case of the node joining Garzón and Sánchez who is assigned to the Sánchez's cluster, or the node joining Garzón and Rajoy who is assigned to Rajoy's cluster. In the plot you can also see two more interesting facts. First, we can see triangle that groups Sánchez, Garzón and Iglesias, which are leaders of the left-wings parties in Spain. Second, some pair of politicians (such as Iglesias-Garzón or Sánchez-Rivera) share more friends than the other possible pairs. 

\pyrex[output=py,format=png,caption=Visualizing the network of Spanish politicians and their followers in Twitter and plotting its clusters]{chapter14/friends2}




